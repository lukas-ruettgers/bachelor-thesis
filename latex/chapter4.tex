% !TEX root = thesis_ruettgers_lukas.tex
% From mitthesis package
% Version: 1.04, 2023/10/19
% Documentation: https://ctan.org/pkg/mitthesis

\chapter{Identifying the Simplest Consistent Algorithm}
We fix a universal reference machine $U$ and consider its induced enumeration of partial computable functions $f_1,f_2,\dots$.
Let $f:\Sigma^{*}\to\Sigma^{*}$ be an arbitrary, partial computable function over an arbitrary, but fixed, finite alphabet $\Sigma$.
For any $D\subseteq \Sigma^{*}$, we define
\begin{align}
	m_U(f\mid D)&:=\min_{\mathcal{S}\subset (D\times \Sigma^{*})^{*}}\{|S|\mid \text{The smallest } f_i \text{ consistent with } \mathcal{S} \text{ satisfies } f_i\equiv f\}, \text{ and}\\
	m_U(f)&:= \min_{D\subseteq \Sigma^{*}} m_U(f\mid D).
\end{align}
Analogously, we consider the amount of information that certainly suffices for all possible training domains and define $\mathcal{F}\subseteq 2^{\Sigma^{*}}$ as the collection of subsets of $\Sigma^{*}$ in which $f$ is still identifiable. That is, for any partial computable function $g\not\equiv f$, $g\lvert_D\not\equiv f\lvert_D$ for all $D\in \mathcal{F}$. Then, we define
\begin{align}
	M_U(f)&:= \max_{D\in\mathcal{F}} m_U(f\mid D).
\end{align}
\section{Out-Sampling Erroneous Simpler Algorithms}
What qualitative and quantitative criteria must the training sample meet to ensure that the simplest algorithm that is consistent with the sample truly coincides with the true function?

\section{Hypothesis Certification}
% see Chapter 5.2 in \cite{li2008kolmogorov}

\section{Retrievability of the Algorithm}
% \cite{richens2024robust} Nearly Optimal Policies for Local Interventions imply Retrievability of Causal Model.
We avail to the idea of \cite{richens2024robust} to derive sufficient conditions when the true algorithm is retrievable.
