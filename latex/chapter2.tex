% !TEX root = thesis_ruettgers_lukas.tex
% From mitthesis package
% Version: 1.04, 2023/10/19
% Documentation: https://ctan.org/pkg/mitthesis


\chapter{Domain Generalization of Neural Networks}
\section{Extrapolation Drawbacks of Models}
\subsection{ReLU MLPs Extrapolate Linearly}
% \cite{hornik1989multilayer} MLPs are universal function approximators
% \cite{xu2019can} ReLU MLP extrapolation in NTK regime
% \cite{jacot2018neural} NTK
Within the support of the training distribution, MLPs are universal function approximators.
In the NTK regime, ReLU MLPs converge to linear functions outside the training distribution with a linear convergence rate.
Instead, the non-linearities in the architecture are the crucial foundation for encoding task-specific non-linearities. Compare Graph Neural Networks and Dynamic Programming Problems.
% TODO: Show own proof outside the NTK regime with regularized ReLU MLPs.

\section{Invariant Causal Mechanisms}
% TODO: At some point we need to introduce the notions of causality.

\section{Optimization Objective Reformulations}
\subsection{Risk Extrapolation}
%\cite{krueger2021out} REx

\subsection{Invariant Risk Minimization}
% TODO: Describe and criticize IRM.
% TODO: Understand the drawback of the linearization of IRM. At what cost did they derive IRMv1?

% \cite{arjovsky2019invariant} IRM
\subsubsection{Fully Informative Invariant Features}
% \cite{ahuja2021invariance} IB-IRM, FIIF vs. PIIF
But I argue that there is a far larger drawback.
\subsubsection{Learning Prime Numbers}
Consider the decision problem \textsc{PrimeNumbers}.
It is widely believed that there is no efficient algorithm that decides prime numbers.
% TODO: Choose citation for hardness of the prime number decision problem.

Empirically, none of the above algorithms perform better than ERM.
