% !TEX root = thesis_ruettgers_lukas.tex
% From mitthesis package
% Version: 1.01, 2023/07/04
% Documentation: https://ctan.org/pkg/mitthesis

\chapter{Secondary Theory and Proofs}

\lstdefinestyle{mystyle}{
	backgroundcolor=\color{CadetBlue!15!white},   
	commentstyle=\color{Red3},
	numberstyle=\tiny\color{gray},
	stringstyle=\color{Blue3},
	basicstyle=\small\ttfamily,
	breakatwhitespace=false,         
	breaklines=true,                 
	numbers=left,                    
	numbersep=5pt,                  
	showspaces=false,                
	showstringspaces=false,
	showtabs=false,                  
	tabsize=2
}%
\lstset{language=C++,style={mystyle}}%
\section{Elementary calculus}
\label{sec:elementary-math}

\begin{lemma}[Log-Linear Inequality]
	\label{lemma:log-lin-inequality}
	Let $a\geq 1$ be an arbitrary real number.
	For any real number $x> 2^{4a}$, $a\log_2(x)< x$.
\end{lemma}
\begin{proof}
	Fix an arbitrary real number $a\geq 1$.
	
	First of all, we show that $x<2^x$ for all $x\in\mathbb{R}$ with $x\geq 1$ by elementary calculus.
	Define the real functions $g(x)=x,h(x)=2^x$.
	Both $g$ and $h$ are continuous differentiable with derivatives $g'(x)=1,h'(x)=\ln(2)2^x$.
	For any $x\geq 1$, the strict monotonicity of the exponential function yields
	\begin{equation}
		\label{eq:exp-lin-derivative-inequality}
		h'(x)=\ln(2)2^x\geq 2\ln(2)=\ln(4)\geq \ln(e)=1=g'(x).
	\end{equation}
	But we also have $h(1)=2>1=g(1)$.
	Therefore, for any $x\geq 1$, it holds that
	\begin{equation}
		\label{eq:exp-lin-inequality}
		2^x=h(x)=h(1)+\int_{1}^{x}h'(x')dx'\overset{(\ref{eq:exp-lin-derivative-inequality})}{>} g(1)+\int_{1}^{x}g'(x')dx'=g(1)+g(x)-g(1)=g(x).
	\end{equation}
	
	In particular, the above result implies $\log_2(x)< x$ for all $x\geq 1$ by substituting $x=2^z$ into Equation \ref{eq:exp-lin-inequality}.
	To generalize this argument to $a\log_2(n)$, we conduct an analogous argument for the real functions $g_a(x)=a\log_2(x)$ and $h(x)=x$.
	
	By $\log_2(x)=\log_2(e)\ln(x)$ for all $x>0$, we obtain the derivative of $g_a$ as $g_a'(x)=\log_2(e)a\frac{1}{x}$.
	Since $\log_2(e) < 2$, any $x\geq 2a$ satisfies
	\begin{equation}
		\label{eq:log-lin-factor-derivative-inequality}
		g'_a(x)=\log_2(e)a\frac{1}{x}< 2a\frac{1}{2a}=1=h'(x).
	\end{equation}
	
	Now, take $x=2^{4a}$. Since $2^{2a} \geq 2^2 = 4$ by the assumption $a\geq 1$, it holds that
	\begin{equation}
		\label{eq:log-lin-factor-inequality-init}
		g_a(x)=a\log_2(x)=a\log_2(2^{4a})=4\cdot a \cdot a \overset{(\ref{eq:exp-lin-inequality})}{<}2^{2a}\cdot 2^a\cdot 2^a=2^{4a}=x=h(x).
	\end{equation}
	
	As $2^{4a}\geq 2^{2a}\geq 2a$, for all $x> 2^{4a}$ this eventually yields
	\begin{align}
		\label{eq:log-lin-inequality}
		x=h(x)&=h(2^{4a})+\int_{2^{4a}}^{x}h'(x')dx'\\
		&\overset{(\ref{eq:log-lin-factor-derivative-inequality})}{>} g_a(2^{4a})+\int_{2^{4a}}^{x}g_a'(x')dx'\\
		&=g_a(2^{4a})+g_a(x)-g_a(2^{4a})=g_a(x)=a\log_2(x).
	\end{align}
\end{proof}

\begin{corollary}[Log-Linear Inequality with Additive Constant]
	\label{cor:log-lin-add-inequality}
	Let $a\geq 1,b\geq 0$ be arbitrary real numbers.
	For any real number $x> 2^{4(a+b)}$, $a\log_2(x)+b < x$.
\end{corollary}
\begin{proof}
	Because $a\geq 1$, it is guaranteed that $x\geq 2^4 \geq 2$, and hence $\log_2(x)\geq 1$. 
	Since $a+b\geq 1$, we employ Lemma \ref{lemma:log-lin-inequality} to conclude $a\log_2(x)+b < (a+b)\log_2(x)\leq x$.
\end{proof}

\begin{lemma}[General Cascade Exponential Inequality]
	\label{lemma:cascade-exp-inequality}
	Let $\exp_2^{(m)}$ denote the $m$-wise recursive concatenation of $\exp_2(x):=2^x$ as in Definition \ref{def:recursive-concatenation}.
	Let $a$ be an arbitrary real scalar with $a\geq 1$, and $k\in\mathbb{N}$. 
	For any natural number $n\in\mathbb{N}$ with $n>4\cdot 2^k\cdot a$ and any real number $x\geq 2$, it holds that
	\begin{equation}
		\exp_2^{(n-k)}(x) > a\cdot n \cdot x.
	\end{equation}
\end{lemma}
\begin{proof}
	Denote by $g_2$ the real function $g_2(x)=2x$.
	A similarly elementary argument as in the proof of Lemma \ref{lemma:log-lin-inequality} yields $\exp_2(x)=2^x\geq 2x=g_2(x)$ for any $x\geq 2$.
	Repeated application of this inequality thence yields 
	\begin{equation}
		\label{eq:multi-exp-2x-inequality}
		\exp^{(n)}_2(x) \geq  g^{(n)}_2(x)= 2^n\cdot x 
	\end{equation}
	for any $n\in\mathbb{N}$.
	In particular, this states that $\exp^{(n-k)}_2(x) \geq 2^{n-k}\cdot x$ for any $n\geq k$.
	
	Now, assume that $n>4\cdot 2^k\cdot a$.
	By Lemma \ref{lemma:log-lin-inequality}, it holds that $2^n> 2^k\cdot a\cdot n$.
	For that reason, we obtain
	\begin{align}
		\label{eq:cascade-exp-inequality}
		\exp^{(n-k)}_2(x) \overset{(\ref{eq:multi-exp-2x-inequality})}{\geq} 2^{n-k}\cdot x\overset{\ref{lemma:log-lin-inequality}}{>} 2^{-k}\cdot 2^k\cdot a\cdot n \cdot x = a\cdot n\cdot x.
	\end{align}
	
\end{proof}
\begin{corollary}[Special Cascade Exponential Equality]
	\label{cor:cascade-exp-inequality-basis}
	Let $a\geq 1,b\geq 0$ be arbitrary real numbers.
	For any $k\in\mathbb{N}$ and $n\in\mathbb{N}$ with $n>4\cdot 2^k\cdot (a+b)$, it holds that
	\begin{equation}
		\exp^{(n-k)}_2(1) > a\cdot n + b.
	\end{equation}
\end{corollary}
\begin{proof}
	Let $a\geq 1,b\geq 0$ be arbitrary real numbers.
	Let $n>4\cdot 2^k \cdot (a+b)$ be arbitrary. 
	First of all, Equation \ref{eq:exp-lin-inequality} asserts that $2^k>k$.
	With $a+b\geq 1$, this implies $n>4\cdot k\geq k+1$ for $k\geq 1$.
	For $k=0$, we similarly have $n>4\cdot 2^k=4>k+1$.
	Therefore, we may reduce $\exp^{(n-k)}_2(1)=\exp^{(n-k-1)}_2(2)$ and conclude in the same fashion as Lemma \ref{lemma:cascade-exp-inequality}:
	\begin{align}
		\exp^{(n-k-1)}_2(2) &\overset{(\ref{eq:multi-exp-2x-inequality})}{\geq} 2^{-k-1}\cdot 2^n \cdot 2\\
		&\overset{(\ref{eq:cascade-exp-inequality})}{>} 2^{-k-1} \cdot 2^k\cdot (a+b)\cdot n \cdot 2 = (a+b)\cdot n \overset{n\geq 1}{\geq} an+b.
	\end{align}
\end{proof}
\section{Inexpressivity of scaled recursive completion}
\label{app:scaled-recursive-completion-inexpressible}
\begin{corollary}[Scaled Recursive Completion is not non-recursively expressible]
	\label{cor:scaled-recursive-completion-not-non-recursive-expressible}
	Let $\tau$ be as in Theorem \ref{theorem:recursive-completion-not-non-recursive-expressible}.
	
	For any $a\in\mathbb{N}$, define the scaled recursive completion over $\tau$ as
	$\left(f_{\tau}\right)_{a}^{\lozenge}(n):=f_{\tau}^{(n)}(a\cdot n)$.
	
	Then, $\left(f_{\tau}\right)_{a}^{\lozenge}\notin \mathcal{F}_{\tau}$ for any $a \geq 1$.
	Let $f\in\mathcal{F}_{\tau}$ be an arbitrary non-recursive function over $\tau$.
	Then, for the same $n_0(f)$ as in Theorem \ref{theorem:recursive-completion-not-non-recursive-expressible}, we have that for all $a\in\mathbb{N}$ with $a\geq 1$ and $n\geq n_0$, $f(n)\neq \left(f_{\tau}\right)_{a}^{\lozenge}(n)$ for all $n\geq n_0(f)$.
\end{corollary}
\begin{proof}
	We show that the thresholds $n_0$ in the proof of Theorem \ref{theorem:recursive-completion-not-non-recursive-expressible} maintain to hold for $\left(f_{\tau}\right)_{s}^{\lozenge}$ for any scale $s\in\mathbb{N}_{\geq 1}$.
	In the following, let $s\in\mathbb{N}$ be arbitrary with $s\geq 1$.
	
	Analogously to Theorem \ref{theorem:recursive-completion-not-non-recursive-expressible}, we prove a stronger argument by induction over the depth of non-recursive functions.
	That is, we show that for all non-recursive functions $f\in\mathcal{F}_{\tau}$, there is an $n_0\in\mathbb{N}$ such that $f(n+a)<f_{\tau}^{(n)}(s\cdot(n+a))$ for all $n\geq n_0$ and all offsets $a\in\mathbb{N}$.
	Note that this statement was already proved for $s=1$ in the original theorem.
	
	Since the assumptions on $\tau$ are the same as in Theorem \ref{theorem:recursive-completion-not-non-recursive-expressible}, we can directly avail to its equations.
	For depth $0$ and any $n\geq n_0=1,a\in\mathbb{N}$, we extend Equations \ref{eq:theorem:recursive-completion-not-non-recursive-expressible-self-lower-bound} and \ref{eq:theorem:recursive-completion-not-non-recursive-expressible-inequality} to
	\begin{align}
		f_{\tau}^{(n)}(s\cdot (n+a)) \overset{(\ref{eq:theorem:recursive-completion-not-non-recursive-expressible-inequality})}{\geq} f_m^{(n)}(s\cdot (n+a))
		\geq f_m^{(n)}(n+a)
		\label{eq:cor-scaled-recursive-completion-not-non-recursive-expressible-self-lower-bound}
		\overset{(\ref{eq:theorem:recursive-completion-not-non-recursive-expressible-self-lower-bound})}{>}n+a  = f(n+a).
	\end{align}
	
	Proceeding with the induction hypothesis (IH), assume that there is some $p\in\mathbb{N}$ such that for every non-recursive function $f\in\mathcal{F}_{\tau}$ with $\operatorname{dep}(f)\leq p$, there exists an $n_0\in\mathbb{N}$ such that $f_\tau^{(n)}(s\cdot(n+a))>f(n+a)$ for all $n\geq n_0$ and $a\in\mathbb{N}$. 
	As in Theorem \ref{theorem:recursive-completion-not-non-recursive-expressible}, we expand any $f\in\mathcal{F}_{\tau}$ with depth $p+1\geq 1$ as $f(n)=f_m(g_1(n),\dots,g_k(n))$ for some $f_m\in\tau$ with arity $k\in\mathbb{N}$ and non-recursive functions $g_i\in \mathcal{F}_{\tau}$ with depth $\operatorname{dep}(g_i)\leq p$ and distinguish by the cases where $f_m$ is bounded or strictly monotonously increasing.
	
	In the case that $f_m$ is bounded by some $c_m$, the same $n_0=c_m$ as in Equation \ref{eq:theorem:recursive-completion-not-non-recursive-expressible-induction-step-bounded} in the proof of the original theorem satisfies 
	\begin{align}
		f_{\tau}^{(n)}(s\cdot (n+a)) \overset{(\ref{eq:cor-scaled-recursive-completion-not-non-recursive-expressible-self-lower-bound})}{>} n+a \geq c_m \geq f(n+a).
	\end{align}
	
	In the other case where $f_m$ is strictly monotonously increasing, we adopt $n_0=\underset{1\leq i\leq k}{\max}n_0(g_i)$ as it stands from the proof of Theorem \ref{theorem:recursive-completion-not-non-recursive-expressible}.
	This $n_0$ is hence independent of $s$.
	Similarly, let $n\geq n_0$ and $a\geq 1$ be arbitrary and substitute $n'=n+1$ and $b=a-1$. Then we analogously have
	\begin{align}
		f(n'+b)=f(n+a)&=f_m(g_1(n+a),\dots,g_k(n+a))\\
		&\overset{(IH)}{<}f_m(f_\tau^{(n)}(s\cdot (n+a)),\dots,f_\tau^{(n)}(s\cdot (n+a)))\\
		&=f_m^{-}(f_\tau^{(n)}(s\cdot (n+a)))\\
		&\overset{\ref{lemma:max-bound-recursive-concatenation-sum}}{\leq} f_\tau^{-}(f_\tau^{(n)}(s\cdot (n+a)))\\
		&=f_{\tau}^{(n+1)}(s\cdot (n+a))=f_{\tau}^{(n')}(s\cdot(n'+b)).
	\end{align}
	
	The overall statement consequently follows by the induction principle, and as a special case, we obtain that for any non-recursive function $f\in\mathcal{F}_{\tau}$, there exists an $n_0(f)\in\mathbb{N}$ such that $f(n)<f_{\tau}^{(n)}(s\cdot n)=\left(f_{\tau}\right)_{s}^{\lozenge}(n)$ for all $n\geq n_0$.
	
\end{proof}

\section{Extension of non-recursive inexpressivity results to integer functions}
\label{app:extension-inexpressivity-integer}
To directly apply our results to neural networks that operate on finite precision floating numbers, we yet need to extend the results of non-recursive functions from Section \ref{sec:expressive-limits-non-recursive-models} to functions sets over $\mathbb{Z}$.

The necessary alterations are only small and the assumptions on the properties of the functions in the function set $\tau$ are still realistic, such that they are directly applicable to multi-layer neural networks and other non-recursive models on common computing architectures.

In particular, the inequality in Equation (\ref{eq:lemma:max-bound-recursive-concatenation-sum-inequality}) in the proof of the lower bound of the sum $f_\tau:=\sum_{i=1}^{j}f_i$ in Lemma \ref{lemma:max-bound-recursive-concatenation-sum} relied on the non-negativity of each $f_i$.
To generalize this statement to functions over $\mathbb{Z}$, we simply take the \textit{magnitude} of each function $f_i$ before summing them up.
Therefore, we redefine the magnitude sum function $f_\tau(n):=\sum_{i=1}^{j}|f_i(n)|$.
Juxtaposing this construction $f_{\tau}$ with non-recursive functions over $\tau$ is still perfectly fair, since both the addition function and the magnitude function fall within the scope of functions that still satisfy our assumptions on $\tau$.
The uniform Turing Machine $\mathcal{T}_{\lozenge}$ from \ref{theorem:recursive-completion-kolmogorov-complexity} only needs a slight adjustment to correctly compute the recursive completion over $\tau$ $f_\tau^{\lozenge}$ according to this new definition.
Representing integers $n\in\mathbb{Z}$ in the two's complement as is usually done in floating point arithmetic on computing architectures, $\mathcal{T}_{\lozenge}$ only has to compute the magnitude of the output after each simulation on the computation tape before it accumulates this output on the accumulation tape.

Secondly, we adapt our assumptions on the functions in $\tau$.
Originally, we assumed that each function $f_i:\mathbb{N}\to\mathbb{N}$ in $\tau$ is either bounded or strictly monotonously increasing. Moreover, we required that at least one strictly monotonously increasing function $f_i$ has an arity $\operatorname{ar}(f_i)>1$.
The definition of bounded function directly transfers to functions $f_i:\mathbb{Z}\to\mathbb{Z}$.
On the other hand, sign flips in functions over $\mathbb{Z}$ sometimes invert the monotonic behaviour of a function, as is the case for the multiplication function.
To reconcile the constraint of strict monotonicity with this behaviour, it is viable to merely require strict monotonicity over natural numbers and bound the magnitude of the function values for non-negative and possibly negative inputs by the function value for the respective non-negative inputs. 
We formalise the latter condition in the following vein
\begin{definition}[Dominance on $\mathbb{N}$]
	\label{def:dominance-on-n}
	Let $f:\mathbb{Z}^{k}\to\mathbb{Z}$ be an arbitrary function.
	We say that $f$ is dominant on $\mathbb{N}$ if for all $x_1,\dots,x_k\in\mathbb{Z}$,
	\begin{align}
		|f(x_1,\dots,x_k)|\leq f(|x_1|,\dots,|x_k|).
	\end{align}
\end{definition}

For at least one strictly monotonously increasing function $f_i\in\tau$, we however still require arity $\operatorname{ar}(f_i)>1$. Additionally, the strict monotonicity of $f_i$ must hold entirely over $\mathbb{Z}$, not only over $\mathbb{N}$ as for the other strictly monotonously increasing functions.
The addition function $f_{+}$ however still satisfies these conditions.
For such functions, we slightly adjust the self-lower bound from Lemma \ref{lemma:self-lower-bound-strictly-monotonous-functions} to take into account the possibly negative function value at $0$.
For completeness, we still provide the full proof.
\begin{lemma}[Lower bound for recursive concatenation of strictly monotonously increasing functions]
	\label{lemma:self-lower-bound-strictly-monotonous-functions-integer}
	Let $f:\mathbb{Z}^{k}\to\mathbb{Z}$ be an arbitrary $k$-ary function that is strictly monotonously increasing with $k\geq 1$.
	Let $f^{(m)}$ denote the $m$-wise recursive concatenation of $f$ as in Definition \ref{def:recursive-concatenation}. 
	% NOTE: It must hold that $k\geq 1$ for strictly monotonously increasing functions.
	
	It holds that $f^{(m)}(n)\geq k^m\cdot n + f^{-}(0)\cdot\sum_{i=0}^{m-1}k^i$ for all $n,m\in\mathbb{N}$.
\end{lemma}
\begin{proof}
	Fix an arbitrary, strictly monotonously increasing function $f$ of arity $k$.
	First of all, we show that $f^{-}(n)\geq k\cdot n + f^{-}(0)$ for all $n\in\mathbb{N}$.
	
	Exhaustive exploitation of strict monotonicity yields
	\begin{align}
		f^{-}(n)=f(\underbrace{n,\dots,n}_{k \text{ times}})&\geq f(n-1,n,\dots,n)+1\\
		&\overset{\dots}{\geq} f(0,n,\dots,n)+n \\
		&\overset{\dots}{\geq} f(0,0,\dots,0)+k\cdot n = f^{-}(0) + k\cdot n.
		\label{eq:self-lower-bound-strictly-monotonous-functions-integer-inequality-depth1}
	\end{align} 
	
	Now, we prove the general lower bound for $f^{(m)}(n)$ by induction over the number of recursive concatenations $m\in\mathbb{N}$.
	
	The base case $m=0$ holds by definition since $f^{(0)}(n)=n=k^0\cdot n + f^{-}(0)\cdot\underbrace{\sum_{i=0}^{-1}k^i}_{=0}$.
	
	For the induction hypothesis (IH), assume that there is some $m\in\mathbb{N}$ such that $f^{(m)}(n)\geq k^m\cdot n + f^{-}(0)\cdot\sum_{i=0}^{m-1}k^i$ holds for all $n\in\mathbb{N}$.
	By the strict monotonicity of $f$ and Equation \ref{eq:self-lower-bound-strictly-monotonous-functions-inequality-depth1}, we obtain
	\begin{align}
		\label{eq:self-lower-bound-strictly-monotonous-functions-integer-inequality-induction-step}
		f^{(m+1)}(n)&=f^{-}(f^{(m)}(n)) \overset{\text{(IH)},(\ref{eq:self-lower-bound-strictly-monotonous-functions-inequality-depth1})}{\geq} k\cdot f^{-}(0)\cdot\sum_{i=0}^{m-1}k^i + f^{-}(0) = k^{m+1} \cdot n + \sum_{i=0}^{(m+1)-1}k^i. 
	\end{align}
	
	Consequently, the overall result follows by the induction principle.
\end{proof}

With the new definition of $f_{\tau}$ that considers the \textit{magnitude} of each function $f_i$, lower bound of the recursive concatenation $f_{\tau}^{(m)}$ may accordingly include the magnitude too.
\begin{lemma}[Lower bound for recursive concatenation of function set sum]
	\label{lemma:max-bound-recursive-concatenation-sum-integer}
	Given an arbitrary function set $\tau=\{f_1,\dots,f_j\},f_i:\mathbb{Z}\to\mathbb{Z}$.
	Denote by $f_{\tau}$ the magnitude sum $f_{\tau}(n)=\sum_{i=1}^{j} \bigl|f_i(n)\bigr|$.
	
	For any monotonously increasing function $f_k\in\tau$ and any $m,n\in\mathbb{N}$, it holds that
	$f_{\tau}^{(m)}(n)\geq \bigl|f_k^{(m)}(n)\bigr|$.
\end{lemma}
\begin{proof}
	Given all as above.
	We show the statement by induction over $m\in\mathbb{N}$.
	
	For $m=0$, the result follows from the definition
	\begin{align*}
		f_{\tau}^{(0)}(n)=\sum_{i=1}^{j} \bigl|f_{i}^{(0)}(n)\bigr|=\sum_{i=1}^{j}n\geq n=f_{k}^{(0)}(n).
	\end{align*}
	for all $1\leq k\leq j$ and $n\in\mathbb{N}$.
	
	Now, assume that for some $m\in\mathbb{N}$, the statement $f_{\tau}^{(m)}(n)\geq |f_i^{(m)}(n)|$ holds for all monotonously increasing functions $f_i\in\tau$ and all $n\in\mathbb{N}$.
	
	For $m+1$, any monotonously increasing $f_k \in \tau$ satisfies
	\begin{align}
		f_{\tau}^{(m+1)}(n) = f_{\tau}^{-}(f_{\tau}^{(m)}(n)) = \sum_{i=1}^{j}\bigl|f_i^{-}(f_{\tau}^{(m)}(n))\bigr|
		&\geq \bigl|f_k^{-}(f_{\tau}^{(m)}(n))\bigr|\\
		&\overset{\text{(IH)}}{\geq} \bigl|f_k^{-}( f_k^{(m)}(n))\bigr| = \bigl|f_k^{(m+1)}(n)\bigr|,
	\end{align}
	where the first inequality draws upon the non-negativity of each summand $\bigl|f_i^{-}(f_{\tau}^{(m)}(n))\bigr|$.
	Therefore, the overall statement for all $m\in\mathbb{N}$ follows by induction.	
	
\end{proof}
With these adjusted bounds at hand, we can obtain the same inexpressivity result as in the original Theorem \ref{theorem:recursive-completion-not-non-recursive-expressible}.
Our requirement that each function in $\tau$ is either bounded, or dominant and strictly monotonously increasing on $\mathbb{N}$ is still satisfied by arithmetic operations such as addition and multiplication, as well as logical expressions and constants.
Moreover, any activation functions that are usually employed in practice can be expressed as finite concatenations of functions that satisfy the above conditions.

It therefore covers the range of functions that are used to construct multi-layer neural networks like MLPs in practice.
\begin{theorem}[Recursive sum completion is not non-recursively expressible]
	\label{theorem:recursive-completion-not-non-recursive-expressible-integer}
	Fix an arbitrary function set $\tau=\{f_1,\dots,f_j\},f_i:\mathbb{Z}^k\to\mathbb{Z}$, where each $f_i$ is either 
	\begin{itemize}
		\item bounded, or
		\item dominant and strictly monotonously increasing on $\mathbb{N}$.
	\end{itemize}
	Assume that there is at least one $f_i\in\tau$ that is strictly monotonously increasing on $\mathbb{Z}$ and has arity $\operatorname{ar}(f_i)>1$, e.g. the addition function.
	For any non-recursive function $f\in\mathcal{F}_{\tau}$, there is an $n_0\in\mathbb{N}$ such that for all $n\geq n_0$, $f(n)\neq f_{\tau}^{\lozenge}(n)$.
\end{theorem}
\begin{proof}
	Given an arbitrary function set $\tau=\{f_1,\dots,f_j\}$ as above.
	Define $f_\tau(n):=\sum_{i=1}^{j}|f_i(n)|$ as in Lemma \ref{lemma:max-bound-recursive-concatenation-sum-integer}.
	
	As it facilitates the induction step, we prove a stronger statement by induction over the depth of non-recursive functions $\operatorname{dep}(f)\in\mathbb{N}$.
	Namely, we are going to prove that for all non-recursive functions $f\in\mathcal{F}_{\tau}$, there is an $n_0\in\mathbb{N}$ such that $|f(n+a)|<f_{\tau}^{(n)}(n+a)$ for all $n\geq n_0$ and all offsets $a\in\mathbb{N}$.
	
	Beginning with the base case, let $f\in\mathcal{F}_{\tau}$ be arbitrary with $\operatorname{dep}(f)=0$.
	Therefore, $f$ must be equivalent to the identify function, thus $f(n)=n$ for all $n\in\mathbb{Z}$.
	
	Let $a\in\mathbb{N}$ be arbitrary.	
	
	Since $k=\operatorname{ar}(f_m)>1$ for at least one strictly monotonously increasing $f_{\ell}\in\tau$, Lemma \ref{lemma:self-lower-bound-strictly-monotonous-functions-integer} guarantees that
	\begin{align}
		\label{eq:theorem:recursive-completion-not-non-recursive-expressible-integer-self-lower-bound-lemma}
		f_{\ell}^{(n)}(n+a) \geq k^n\cdot (n+a) + f_{\ell}^{-}(0)\cdot\sum_{i=0}^{n-1}k^i. %\overset{k>1}{>} n+a.
	\end{align}
	for any $n\in\mathbb{N}$.
	Because $k>1$, it further holds by the geometric sum that
	\begin{align}
		\label{eq:theorem:recursive-completion-not-non-recursive-expressible-integer-geom-sum-bound}
		f_{\ell}^{-}(0)\cdot\sum_{i=0}^{n-1}k^i & \geq  -\bigl|f_{\ell}^{-}(0)\bigr|\cdot \frac{k^n-1}{k-1} \geq -\bigl|f_{\ell}^{-}(0)\bigr|\cdot(k^n-1).
	\end{align}
	
	Choose $n_0:=|f_{\ell}^{-}(0)|+1$ and let $n\geq n_0$ be arbitrary.
	Joining the above inequalities, we obtain
	\begin{align}
		\label{eq:theorem:recursive-completion-not-non-recursive-expressible-integer-self-lower-bound}
		f_{\ell}^{(n)}(n+a) &\overset{(\ref{eq:theorem:recursive-completion-not-non-recursive-expressible-integer-self-lower-bound-lemma})}{\geq} k^n\cdot (n+a) + f_{\ell}^{-}(0)\cdot\sum_{i=0}^{n-1}k^i\\
		&\overset{(\ref{eq:theorem:recursive-completion-not-non-recursive-expressible-integer-geom-sum-bound})}{\geq} k^n\cdot (n+a) -(k^n-1)\cdot \bigl|f_{\ell}^{-}(0)\bigr| \\
		&= (k^n-1)\cdot\bigl(\underbrace{n+a-\bigl|f_{\ell}^{-}(0)\bigr|}_{> 0}\bigr) + n+a\\
		&> n+a = |n+a|.
	\end{align}
	
	Since $f_{\ell}$ is monotonously increasing, we conclude by means of Lemma \ref{lemma:max-bound-recursive-concatenation-sum-integer} that
	\begin{equation}
		\label{eq:theorem:recursive-completion-not-non-recursive-expressible-integer-inequality}
		f_{\tau}^{(n)}(n+a)\overset{\ref{lemma:max-bound-recursive-concatenation-sum-integer}}{\geq} f_{\ell}^{(n)}(n+a) \overset{(\ref{eq:theorem:recursive-completion-not-non-recursive-expressible-integer-self-lower-bound})}{>} |n+a| = |f(n+a)|.
	\end{equation}
	
	Proceeding with the induction hypothesis (IH), assume that there is some $p\in\mathbb{N}$ such that for every non-recursive function $f\in\mathcal{F}_{\tau}$ with $\operatorname{dep}(f)\leq p$, there exists an $n_0\in\mathbb{N}$ such that $f_\tau^{(n)}(n+a)>f(n+a)$ for all $n\geq n_0$ and $a\in\mathbb{N}$.
	
	Let $a\in\mathbb{N}$ be arbitrary in the following.
	Consider an arbitrary non-recursive function $f\in\mathcal{F}_{\tau}$ with $\operatorname{dep}(f)=p+1$.
	
	As by Definition \ref{def:non-recursive-functions}, we have $f(n)=f_m(g_1(n),\dots,g_k(n))$ for some $f_m\in\tau$ with arity $k\in\mathbb{N}$ and non-recursive functions $g_i\in \mathcal{F}_{\tau}$ with depth $\operatorname{dep}(g_i)\leq p$.
	
	To begin with, we consider the case where $f_m$ is a bounded function. Denote by $c_m$ the bound of $f_m$, that is $|f_m(n)|\leq c_m$ for all $n\in\mathbb{Z}$.
	For any $n\geq \max\bigl(|f_{\ell}^{-}(0)|+1,c_m\bigl)$, where $f_{\ell}$ is the strictly monotonously increasing function from before, Equation \ref{eq:theorem:recursive-completion-not-non-recursive-expressible-inequality} assures that
	\begin{equation}
		\label{eq:theorem:recursive-completion-not-non-recursive-expressible-induction-step-bounded}
		f_\tau^{(n)}(n+a)
		\overset{(\ref{eq:theorem:recursive-completion-not-non-recursive-expressible-inequality})}{>} n+a\geq c_m\geq \bigl|f_m\bigl(g_1(n+a),\dots,g_k(n+a)\bigr)\bigr|=|f(n+a)|.
	\end{equation}
	
	Otherwise, consider the case where $f_m$ is dominant and strictly monotonously increasing on $\mathbb{N}$.
	By the induction hypothesis, for every non-recursive function $g_i$, there is an $n_0(g_i)$ such that $f_\tau^{(n)}(n+a)>|g_i(n+a)|$ for all $n\geq n_0(g_i)$.
	We hence directly take the threshold as the maximum $n_0 = \max_{1\leq i\leq k}n_0(g_i)$.
	
	Now, let $n\geq n_0$ be arbitrary. 
	Subsequently, we assume that $a\geq 1$, and substitute $b=a-1$ and $n'=n+1$.
	By the strict monotonicity of $f_m$ on $\mathbb{N}$, we conclude
	\begin{align}
		\label{eq:theorem:recursive-completion-not-non-recursive-expressible-integer-induction-step}
		|f(n'+b)|=|f(n+a)|
		&=\bigl|f_m\bigl(g_1(n+a),\dots,g_k(n+a)\bigr)\bigr|\\
		&\overset{\ref{def:dominance-on-n}}{\leq} f_m\bigl(|g_1(n+a)|,\dots,|g_k(n+a)|\bigr)\\
		&\overset{(IH)}{<}f_m(f_\tau^{(n)}(n+a),\dots,f_\tau^{(n)}(n+a))\\
		&\leq\bigl|f_m^{-}(f_\tau^{(n)}(n+a))\bigr|\\
		&\leq f_\tau^{-}(f_\tau^{(n)}(n+a))=f_{\tau}^{(n+1)}(n+a)=f_{\tau}^{(n')}(n'+b).
	\end{align}
	
	Since $a\in\mathbb{N}_{\geq 1}$ was chosen arbitrarily, the above argument holds for all $b\in\mathbb{N}$ and all $n'\geq n_0+1=:n_0(f)$. 
	To conclude with, we have shown that for any non-recursive function $f\in \mathcal{F}_{\tau}$ with $\operatorname{dep}(f)=p+1$, there exists an $n_0(f)\in\mathbb{N}$ such that $|f(n+a)|<f_\tau^{(n)}(n+a)$ for all $n\geq n_0(f),a\in\mathbb{N}$.
	
	Consequently, the overall result for non-recursive functions of arbitrary depth draws upon the induction principle.
	
	As a special case of the above result, we obtain that for any non-recursive function $f\in \mathcal{F}_{\tau}$, there is an $n_0\in\mathbb{N}$ such that for all $n\geq n_0$, $f(n)<f_{\tau}^{\lozenge}(n)$.
	
\end{proof}