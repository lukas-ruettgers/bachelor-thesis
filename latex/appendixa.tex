% !TEX root = thesis_ruettgers_lukas.tex
% From mitthesis package
% Version: 1.01, 2023/07/04
% Documentation: https://ctan.org/pkg/mitthesis

\chapter{Secondary Theory and Proofs}

\lstdefinestyle{mystyle}{
	backgroundcolor=\color{CadetBlue!15!white},   
	commentstyle=\color{Red3},
	numberstyle=\tiny\color{gray},
	stringstyle=\color{Blue3},
	basicstyle=\small\ttfamily,
	breakatwhitespace=false,         
	breaklines=true,                 
	numbers=left,                    
	numbersep=5pt,                  
	showspaces=false,                
	showstringspaces=false,
	showtabs=false,                  
	tabsize=2
}%
\lstset{language=C++,style={mystyle}}%
\section{Elementary calculus}
\label{sec:elementary-math}

\begin{lemma}[Log-Linear Inequality]
	\label{lemma:log-lin-inequality}
	Let $a\geq 1$ be an arbitrary real number.
	For any real number $x> 2^{4a}$, $a\log_2(x)< x$.
\end{lemma}
\begin{proof}
	Fix an arbitrary real number $a\geq 1$.
	
	First of all, we show that $x<2^x$ for all $x\in\mathbb{R}$ with $x\geq 1$ by elementary calculus.
	Define the real functions $g(x)=x,h(x)=2^x$.
	Both $g$ and $h$ are continuous differentiable with derivatives $g'(x)=1,h'(x)=\ln(2)2^x$.
	For any $x\geq 1$, the strict monotonicity of the exponential function yields
	\begin{equation}
		\label{eq:exp-lin-derivative-inequality}
		h'(x)=\ln(2)2^x\geq 2\ln(2)=\ln(4)\geq \ln(e)=1=g'(x).
	\end{equation}
	But we also have $h(1)=2>1=g(1)$.
	Therefore, for any $x\geq 1$, it holds that
	\begin{equation}
		\label{eq:exp-lin-inequality}
		2^x=h(x)=h(1)+\int_{1}^{x}h'(x')dx'\overset{(\ref{eq:exp-lin-derivative-inequality})}{>} g(1)+\int_{1}^{x}g'(x')dx'=g(1)+g(x)-g(1)=g(x).
	\end{equation}
	
	In particular, the above result implies $\log_2(x)< x$ for all $x\geq 1$ by substituting $x=2^z$ into Equation \ref{eq:exp-lin-inequality}.
	To generalize this argument to $a\log_2(n)$, we conduct an analogous argument for the real functions $g_a(x)=a\log_2(x)$ and $h(x)=x$.
	
	By $\log_2(x)=\log_2(e)\ln(x)$ for all $x>0$, we obtain the derivative of $g_a$ as $g_a'(x)=\log_2(e)a\frac{1}{x}$.
	Since $\log_2(e) < 2$, any $x\geq 2a$ satisfies
	\begin{equation}
		\label{eq:log-lin-factor-derivative-inequality}
		g'_a(x)=\log_2(e)a\frac{1}{x}< 2a\frac{1}{2a}=1=h'(x).
	\end{equation}
	
	Now, take $x=2^{4a}$. Since $2^{2a} \geq 2^2 = 4$ by the assumption $a\geq 1$, it holds that
	\begin{equation}
		\label{eq:log-lin-factor-inequality-init}
		g_a(x)=a\log_2(x)=a\log_2(2^{4a})=4\cdot a \cdot a \overset{(\ref{eq:exp-lin-inequality})}{<}2^{2a}\cdot 2^a\cdot 2^a=2^{4a}=x=h(x).
	\end{equation}
	
	As $2^{4a}\geq 2^{2a}\geq 2a$, for all $x> 2^{4a}$ this eventually yields
	\begin{align}
		\label{eq:log-lin-inequality}
		x=h(x)&=h(2^{4a})+\int_{2^{4a}}^{x}h'(x')dx'\\
		&\overset{(\ref{eq:log-lin-factor-derivative-inequality})}{>} g_a(2^{4a})+\int_{2^{4a}}^{x}g_a'(x')dx'\\
		&=g_a(2^{4a})+g_a(x)-g_a(2^{4a})=g_a(x)=a\log_2(x).
	\end{align}
\end{proof}

\begin{corollary}[Log-Linear Inequality with Additive Constant]
	\label{cor:log-lin-add-inequality}
	Let $a\geq 1,b\geq 0$ be arbitrary real numbers.
	For any real number $x> 2^{4(a+b)}$, $a\log_2(x)+b < x$.
\end{corollary}
\begin{proof}
	Because $a\geq 1$, it is guaranteed that $x\geq 2^4 \geq 2$, and hence $\log_2(x)\geq 1$. 
	Since $a+b\geq 1$, we employ Lemma \ref{lemma:log-lin-inequality} to conclude $a\log_2(x)+b < (a+b)\log_2(x)\leq x$.
\end{proof}

\begin{lemma}[General Cascade Exponential Inequality]
	\label{lemma:cascade-exp-inequality}
	Let $\exp_2^{(m)}$ denote the $m$-wise recursive concatenation of $\exp_2(x):=2^x$ as in Definition \ref{def:recursive-concatenation}.
	Let $a$ be an arbitrary real scalar with $a\geq 1$, and $k\in\mathbb{N}$. 
	For any natural number $n\in\mathbb{N}$ with $n>4\cdot 2^k\cdot a$ and any real number $x\geq 2$, it holds that
	\begin{equation}
		\exp_2^{(n-k)}(x) > a\cdot n \cdot x.
	\end{equation}
\end{lemma}
\begin{proof}
	% TODO: Replace $\bigcirc$ notation by the $m$-wise recursive concatenation.
	Denote by $g_2$ the real function $g_2(x)=2x$.
	A similarly elementary argument as in the proof of Lemma \ref{lemma:log-lin-inequality} yields $\exp_2(x)=2^x\geq 2x=g_2(x)$ for any $x\geq 2$.
	Repeated application of this inequality thence yields 
	\begin{equation}
		\label{eq:multi-exp-2x-inequality}
		\left(\bigcirc_{i=1}^n\exp_2\right)(x) \geq \left(\bigcirc_{i=1}^n g_2\right)(x)= 2^n\cdot x \quad \text{ for any } n\in\mathbb{N}.
	\end{equation}
	
	In particular, this states that $\left(\bigcirc_{i=1}^{n-k}\exp_2\right)(x) \geq 2^{n-k}\cdot x$ for any $n\geq k$.
	
	Now, assume that $n>4\cdot 2^k\cdot a$.
	By Lemma \ref{lemma:log-lin-inequality}, it holds that $2^n> 2^k\cdot a\cdot n$.
	For that reason, we obtain
	\begin{align}
		\label{eq:cascade-exp-inequality}
		\left(\bigcirc_{i=1}^{n-k}\exp_2\right)(x) &\overset{(\ref{eq:multi-exp-2x-inequality})}{\geq} 2^{n-k}\cdot x\\
		&\overset{\ref{lemma:log-lin-inequality}}{>} 2^{-k}\cdot 2^k\cdot a\cdot n \cdot x = a\cdot n\cdot x.
	\end{align}
	
\end{proof}
\begin{corollary}[Special Cascade Exponential Equality]
	\label{cor:cascade-exp-inequality-basis}
	Let $a\geq 1,b\geq 0$ be arbitrary real numbers.
	For any $k\in\mathbb{N}$ and $n\in\mathbb{N}$ with $n>4\cdot 2^k\cdot (a+b)$, it holds that
	\begin{equation}
		\left(\bigcirc_{i=1}^{n-k}\exp_2\right)(1) > a\cdot n + b.
	\end{equation}
\end{corollary}
\begin{proof}
	Let $a\geq 1,b\geq 0$ be arbitrary real numbers.
	Let $n>4\cdot 2^k \cdot (a+b)$ be arbitrary. 
	First of all, Equation \ref{eq:exp-lin-inequality} asserts that $2^k>k$.
	With $a+b\geq 1$, this implies $n>4\cdot k\geq k+1$ for $k\geq 1$.
	For $k=0$, we similarly have $n>4\cdot 2^k=4>k+1$.
	Therefore, we may reduce $\left(\bigcirc_{i=1}^{n-k}\exp_2\right)(1)=\left(\bigcirc_{i=1}^{n-k-1}\exp_2\right)(2)$ and conclude in the same fashion as Lemma \ref{lemma:cascade-exp-inequality}:
	\begin{align}
		\left(\bigcirc_{i=1}^{n-k-1}\exp_2\right)(2) &\overset{(\ref{eq:multi-exp-2x-inequality})}{\geq} 2^{-k-1}\cdot 2^n \cdot 2\\
		&\overset{(\ref{eq:cascade-exp-inequality})}{>} 2^{-k-1} \cdot 2^k\cdot (a+b)\cdot n \cdot 2\\
		& = (a+b)\cdot n \overset{n\geq 1}{\geq} an+b.
	\end{align}
\end{proof}
\section{Inexpressivity of scaled recursive completion}
\label{app:scaled-recursive-completion-inexpressible}
\begin{corollary}[Scaled Recursive Completion is not non-recursively expressible]
	\label{cor:scaled-recursive-completion-not-non-recursive-expressible}
	Let $\tau$ be as in Theorem \ref{theorem:recursive-completion-not-non-recursive-expressible}.
	
	For any $a\in\mathbb{N}$, define the scaled recursive completion over $\tau$ as
	$\left(f_{\tau}\right)_{a}^{\lozenge}(n):=f_{\tau}^{(n)}(a\cdot n)$.
	
	Then, $\left(f_{\tau}\right)_{a}^{\lozenge}\notin \mathcal{F}_{\tau}$ for any $a \geq 1$.
	Let $f\in\mathcal{F}_{\tau}$ be an arbitrary non-recursive function over $\tau$.
	Then, for the same $n_0(f)$ as in Theorem \ref{theorem:recursive-completion-not-non-recursive-expressible}, we have that for all $a\in\mathbb{N}$ with $a\geq 1$ and $n\geq n_0$, $f(n)\neq \left(f_{\tau}\right)_{a}^{\lozenge}(n)$ for all $n\geq n_0(f)$.
\end{corollary}
\begin{proof}
	We show that the thresholds $n_0$ in the proof of Theorem \ref{theorem:recursive-completion-not-non-recursive-expressible} maintain to hold for $\left(f_{\tau}\right)_{s}^{\lozenge}$ for any scale $s\in\mathbb{N}_{\geq 1}$.
	In the following, let $s\in\mathbb{N}$ be arbitrary with $s\geq 1$.
	
	Analogously to Theorem \ref{theorem:recursive-completion-not-non-recursive-expressible}, we prove a stronger argument by induction over the depth of non-recursive functions.
	That is, we show that for all non-recursive functions $f\in\mathcal{F}_{\tau}$, there is an $n_0\in\mathbb{N}$ such that $f(n+a)<f_{\tau}^{(n)}(s\cdot(n+a))$ for all $n\geq n_0$ and all offsets $a\in\mathbb{N}$.
	Note that this statement was already proved for $s=1$ in the original theorem.
	
	Since the assumptions on $\tau$ are the same as in Theorem \ref{theorem:recursive-completion-not-non-recursive-expressible}, we can directly avail to its equations.
	For depth $0$ and any $n\geq n_0=1,a\in\mathbb{N}$, we extend Equations \ref{eq:theorem:recursive-completion-not-non-recursive-expressible-self-lower-bound} and \ref{eq:theorem:recursive-completion-not-non-recursive-expressible-inequality} to
	\begin{align}
		f_{\tau}^{(n)}(s\cdot (n+a)) &\overset{(\ref{eq:theorem:recursive-completion-not-non-recursive-expressible-inequality})}{\geq} f_m^{(n)}(s\cdot (n+a))\\
		&\geq f_m^{(n)}(n+a)\\
		\label{eq:cor-scaled-recursive-completion-not-non-recursive-expressible-self-lower-bound}
		&\overset{(\ref{eq:theorem:recursive-completion-not-non-recursive-expressible-self-lower-bound})}{>}n+a  = f(n+a).
	\end{align}
	
	Proceeding with the induction hypothesis (IH), assume that there is some $p\in\mathbb{N}$ such that for every non-recursive function $f\in\mathcal{F}_{\tau}$ with $\operatorname{dep}(f)\leq p$, there exists an $n_0\in\mathbb{N}$ such that $f_\tau^{(n)}(s\cdot(n+a))>f(n+a)$ for all $n\geq n_0$ and $a\in\mathbb{N}$. 
	As in Theorem \ref{theorem:recursive-completion-not-non-recursive-expressible}, we expand any $f\in\mathcal{F}_{\tau}$ with depth $p+1\geq 1$ as $f(n)=f_m(g_1(n),\dots,g_k(n))$ for some $f_m\in\tau$ with arity $k\in\mathbb{N}$ and non-recursive functions $g_i\in \mathcal{F}_{\tau}$ with depth $\operatorname{dep}(g_i)\leq p$ and distinguish by the cases where $f_m$ is bounded or strictly monotonously increasing.
	
	In the case that $f_m$ is bounded by some $c_m$, the same $n_0=c_m$ as in Equation \ref{eq:theorem:recursive-completion-not-non-recursive-expressible-induction-step-bounded} in the proof of the original theorem satisfies $f_{\tau}^{(n)}(s\cdot (n+a)) \overset{(\ref{eq:cor-scaled-recursive-completion-not-non-recursive-expressible-self-lower-bound})}{>} n+a \geq c_m \geq f(n+a)$.
	
	In the other case where $f_m$ is strictly monotonously increasing, we adopt $n_0=\max_{1\leq i\leq k} n_0(g_i)$ as it stands from the proof of Theorem \ref{theorem:recursive-completion-not-non-recursive-expressible}.
	Similarly, let $n\geq n_0$ and $a\geq 1$ be arbitrary and substitute $n'=n+1$ and $b=a-1$. Then we analogously have
	\begin{align}
		f(n'+b)=f(n+a)&=f_m(g_1(n+a),\dots,g_k(n+a))\\
		&\overset{(IH)}{<}f_m(f_\tau^{(n)}(s\cdot (n+a)),\dots,f_\tau^{(n)}(s\cdot (n+a)))=f_m^{-}(f_\tau^{(n)}(s\cdot (n+a)))\\
		&\overset{\ref{lemma:max-bound-recursive-concatenation-sum}}{\leq} f_\tau^{-}(f_\tau^{(n)}(s\cdot (n+a)))=f_{\tau}^{(n+1)}(s\cdot (n+a))=f_{\tau}^{(n')}(s\cdot(n'+b)).
	\end{align}
	
	The overall statement consequently follows by the induction principle, and as a special case, we obtain that for any non-recursive function $f\in\mathcal{F}_{\tau}$, there exists an $n_0(f)\in\mathbb{N}$ such that $f(n)<f_{\tau}^{(n)}(s\cdot n)=\left(f_{\tau}\right)_{s}^{\lozenge}(n)$ for all $n\geq n_0$.
	
	This $n_0$ does not depend on $s$ and applies for all $s\geq 1$, which proves our result.
	
\end{proof}