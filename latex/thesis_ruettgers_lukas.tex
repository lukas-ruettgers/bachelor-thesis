% !TeX document-id = {7fb7105e-e482-4fa1-9dae-86c0830318d1}
% !TEX encoding = UTF-8 Unicode
% !BIB TS-program = biber 
% !BIB program = biber    

% This file is MIT-Thesis.tex, a LaTeX template for formatting an MIT thesis with the mitthesis class.
%
% Version: 1.11, 2023/11/02
%
% Author: John H. Lienhard, copyright 2023. Reuse under the MIT license: https://ctan.org/license/mit 

% Documentation is here: https://ctan.org/pkg/mitthesis

%% Don't modify the \DocumentMetadata command unless you know what it does. 
%% If this command throws an "undefined" error, your latex system is out of date: try commenting this command out.
\DocumentMetadata{ 
	pdfstandard = a-2b,
	pdfversion  = 1.7,
	lang		= en-US,
%	debug		= {xmp-export}, % uncomment to output a separate xmpi file showing the metadata
}
%%%%%%%%%%%%%%%%%%%%%%%%%%%%%%%%%%%%%%%

\documentclass{mitthesis} %,fontset=libertine, fontset=newtx-sans-text, fontset=heros-stix2, fontset=stix2
% Use option two-sided for two-sided printing. \documentclass[two-sided]{mitthesis}
%
% option [twoside]		gives facing-page behavior for printing; omitting twoside will eliminate even-numbered blank pages.
% option [lineno]	 	provides line numbers, as for editing
% option [mydesign] 	loads packages for color, title and list formats, margins, or captions: edit mydesign.tex to change defaults.
% option [fontset] is a keyvalue which can be:
%					 	pdftex or unicode engines:  defaultfonts, libertine, lucida
%					 	pdftex only: 				fira-newtxsf, newtx, newtx-sans-text
%						unicode engines (luatex):	heros-stix2, stix2, termes, termes-stix2
%					 	if no key value is given, fonts default to CMR (pdftex) or LMR (unicode), i.e., "the LaTeX font".
%					 	You can edit the fontset files or you can write your own, myfonts.tex, and do [fontset=myfonts].
%						If you are using multiple languages, load the babel package in your fontset file, before the fonts.

%%%%%%%%% Packages used in sample chapters (not otherwise required) %%%%%%%
%% Math packages
\usepackage{amsmath}
\usepackage{amssymb}

%% Package for code listing in Appendix A.
\usepackage{listings}%   documentation is here https://ctan.org/pkg/listings

%% Latin filler used in Chapter 1, with a test for package version date. https://ctan.org/pkg/lipsum
\usepackage{lipsum}
\IfPackageAtLeastTF{lipsum}{2021/09/20}{\setlipsum{auto-lang=false}}{}


%%%%%%%%%  Graphics path (to figure files)  %%%%%%%%%%%%%%%%%%%%%%%%%%%%%%%%

%% Can set graphicspath to point to specific directories containing figures (the current directory is searched automatically)
%% For instance, to search a subdirectory of the current directory called "figures" and a parallel directory called "art", set:

% \graphicspath{ {figures/} {../art/} }% For details see: https://latexref.xyz/dev/latex2e.html#g_t_005cgraphicspath


%%%%%%%%%  Representative set-up for biblatex  %%%%%%%%%%%%%%%%%%%%%%%%%%%%%

\usepackage[style=ieee,maxbibnames=10,sorting=none]{biblatex}% style=ext-numeric-comp,articlein=false,giveninits=true
	\DefineBibliographyStrings{english}{url= \textsc{url} ,  }% replaces default "[Online]. Available" by "URL"


\addbibresource{bib/mainref.bib}

%% to avoid split urls and stretched white space, you can set the bibliography ragged-right:
%\appto{\bibsetup}{\raggedright}

% biblatex is very powerful, and you can customize most aspects the reference list and citations to suit your needs.
% documentation is here: https://ctan.org/pkg/biblatex


%%%%%%%%%%  Option to use natbib   %%%%%%%%%%%%%%%%%%%%%%%%%%%%%%%%%%%%%%%%%

%\RequirePackage[numbers,sort&compress]{natbib}
 
%%% add bibliography to table of contents
%\apptocmd{\bibliography}{\addcontentsline{toc}{chapter}{\protect\textbf{\bibname}}}{}{}

%%% You can use this to rename the bibliography section
%\renewcommand{\bibname}{References}

%%% Can adjust space between bibliography items (change 4pt to something else; don't drop last two lengths, they are stretchable "glue")
%\setlength\bibsep{4pt plus 1pt minus 1pt}


%%%%%%%%%%  Table related packages  %%%%%%%%%%%%%%%%%%%%%%%%%%%%%%%%%%%%%%%%

\usepackage{booktabs}% better quality tables, https://ctan.org/pkg/booktabs
\usepackage{array}%    additional options for table columns, https://ctan.org/pkg/array

%\usepackage{tabularx}%   https://ctan.org/pkg/tabularx

%\usepackage{dcolumn}%    alignment on decimal place, https://ctan.org/pkg/dcolumn
%\newcolumntype{d}[1]{D{.}{.}{#1}}

%%%%%%%%%%  Custom packages  %%%%%%%%%%%%%%%%%%%%%%%%%%%%%%%%%%%%%%%%%%%%%%%


%%%%%%%%%%  Option for "double spacing" %%%%%%%%%%%%%%%%%%%%%%%%%%%%%%%%%%%%

%% Back in the typewriter era, double spaced lines were convenient for editing with a pencil. 
%% In typography, the separation between lines is called "leading", and it is usually set in 
%% proportion to the font size (i.e., when the font is loaded).  If you really feel the need 
%% to change the line separation, the most attractive results will be obtained by changing the
%% leading in proportion to the the current font size, rather than just doubling the space.

%% The setspace package provides a tool for changing line separation. Use these two commands here:
%
% \usepackage{setspace}%  documentation at https://ctan.org/pkg/setspace
% \setstretch{1.1}% you can choose some other value for the stretch of space between lines
%
%% Use one or more of the these commands AFTER the frontmatter
%
% \onehalfspacing
% \doublespacing
% \singlespacing  % will turn these effects off (you can use these anywhere in the document)

%% The best result may be to stay with leading selected by the typographer who set up the font.


%%%%%%%%%%%  Metadata  %%%%%%%%%%%%%%%%%%%%%%%%%%%%%%%%%%%%%%%%%%%%%%%%%%%%%%%

% Most of the document metadata is created automatically. 
% The following items should be adjusted to match your work. <================= !!!!!!!!!!

\hypersetup{%
	pdfsubject={Learning by detecting recursion: single-source domain extrapolation for numeral system data},
	% Change this to briefly state topic of your thesis 
% 
	pdfkeywords={RWTH, THU, Machine Learning, Theory, Deep Learning, Generalization, single source, out of distribution, Ruettgers, Rüttgers, Lukas, bachelor, thesis},
	% Add keywords that will help search engines and libraries to find your work.
	% Includes the name[s] of the author[s] 
	% (If you have used \DocumentMetadata, at line 15, you can just put "\CopyrightAuthor," for the names.)
%
	pdfurl={},
	% If you have a url for the thesis, put it here. Otherwise delete this.
	% (MIT Libraries will put your thesis in DSPACE with a persistent url after you submit it.)
%	
	pdfcontactemail={lukas.ruettgers1@rwth-aachen.de},
	% You can put a [permanent] email address into the metadata, if you like.
	% Otherwise delete this.
%
	pdfauthortitle={},
	% If you have a title, you can include it here.
}

%%%%%%%%%%%%%%  End preamble %%%%%%%%%%%%%%%%%%%%%%%%%%%%%%%%%%%%%%%%%%%%%%%%%%%%%%%%%%%%%%%%%%%%%
%%%%%%%%%%%%%%%%%%%%%%%%%%%%%%%%%%%%%%%%%%%%%%%%%%%%%%%%%%%%%%%%%%%%%%%%%%%%%%%%%%%%%%%%%%%%%%%%%%

\begin{document}

%%% edit the following commands to match your thesis %%%%%%%%%%

\title{Unlocking Recursion: On Out-of-Distribution Generalization with a Simple but Powerful Principle}
% Alternative titles: 
% - Information Thresholds for the Learnability of Partial Computable Functions
% - 
% - Optimal sufficient information for Hypothesis Certification
% - Simple to describe, hard to compute: Generalize beyond Training Domains by Learning Recursion
% TODO: Decide title

% \Author{Author full name}{Author department}[Author's first PREVIOUS degree][Author's second PREVIOUS degree][...
% Note that third, fourth, fifth, and sixth arguments are optional [] and may be omitted

% note on names: most of the following names are made up; Silas Holman was a physics professor at MIT in the 19th century.

\Author{Lukas J. Rüttgers}{Department of Computer Science}
% \Author{Luisa Hernández}{Department of Research}[B.S. Mechanical Engineering, UCLA, 2018][M.S. Stellar Interiors, Vulcan Science Academy, 2020]


% Use once for each degree fulfilled by thesis
% For two degrees from one department, leave the department argument blank for the second degree {}.
% \Degree{Bachelor of Science in Physics}{Department of Physics}
% \Degree{Master of Science in Physics}{}
\Degree{Bachelor of Science in Computer Science}{Department of Computer Science}

% If there is more than one supervisor, use the \Supervisor command for each.
\Supervisor{Jingzhao Zhang}{Assistant Professor of Computer Science, Tsinghua University}
\Supervisor{Hector Geffner}{Professor of Computer Science, RWTH Aachen University}
% \Supervisor{Secunda Castor}{Professor of Research}
% \Supervisor{Quintus Castor}{Professor of Log Dams}

% Professor who formally accepts theses for your department (e.g., the Graduate Officer, Professor Sméagol,...)
% If more than one department, use more than once
% **If you need to reduce vertical space, put the acceptor title in the second argument and leave the third blank {}.**
% \Acceptor{Primus Castor}{Professor of Wetlands Engineering}{Undergraduate Officer, Department of Physics}
% \Acceptor{Tertius Castor}{Professor of Log Dams}{Graduate Officer, Department of Research}
% \Acceptor{Quarta Castor}{Professor of Lodge Building}{Graduate Officer, Department of Mechanical Engineering}

% Usage: \DegreeDate{Month}{year}
% Valid degree months are September, February, or June
\DegreeDate{September}{2024}

% Date that final thesis is submitted to department
\ThesisDate{June 30, 2024}

\Institution{RWTH Aachen University}

%%%%%%  Choose whether to have a CREATIVE COMMONS License  %%%%%%%%%%%%%%%%%%%%%%%%%%%%%%%%%%%%%%
%
% If you are using a cc license, put details of your cc license here. 
% Omit this command if you are not using a cc license.
%
\CClicense{CC BY-NC-ND 4.0}{https://creativecommons.org/licenses/by-nc-nd/4.0/}
%

%%%%%%%  Solutions for overflowing titlepage  %%%%%%%%%%%%%%%%%%%%%%%%%%%%%%%%%%%%%%%%%%%%%%%%%%%

% If your title page is overflowing (from too many names, degrees, etc.):
%
% (a) you can reduce the 12pt and 18pt skips between various blocks to 6pt with this command:
%
% \Tighten
%
% (b)  you can scale down the Signature block at the bottom with this command:
%
% \SignatureBlockSize{\small}  %or this one \SignatureBlockSize{\footnotesize}
%
% (c) you can put the acceptor name and title onto two lines, rather than three like this:
%
% \Acceptor{Tertius Castor}{Professor and Graduate Officer, Department of Research}{}
% \Acceptor{Quarta Castor}{Professor and Graduate Officer, Department of Mechanical Engineering}{}
%
% (d) you can change the font size of the the author name[s] with
%
%	\AuthorNameSize{\normalsize}
%
% (e) and you can omit any previous degrees from the title page, instead mentioning them in the Biosketch

% Also, if you prefer to keep the text toward the top of the page with most white space at the bottom, you
% can you this command to squash all of the vertical glue (stretchy space) with this command:
%
% \Squash 
%
% This command is useful when the text has not already reach the bottom of the page, since the glue gets squashed automatically
% when the page is too full.

%%%%%%%%%%%%%%%%%%%%%%%%%%%%%%%%%%%%%%%%%%%%%%%%%%%%%%%%%%%%%%%%%%%%%%%%%%%%%%%%%%%%%%%%%%%%%%%%%

%%% Make titlepage
\SuppressAcceptorError % There is no acceptor at RWTH.
\maketitle*

%%%%%%%%% Contents that you need to write follows %%%%%%%%%%%%%%%%%%%%%%%%%%%%%%%%%%%%%%%%%%%%%%%%

% \includeonly{acknowledgments,biography,chapter1,chapter2,...,appendixa,...} 
%   for usage, see https://latexref.xyz/_005cinclude-_0026-_005cincludeonly.html

%%% Frontmatter (write this material in the mentioned files)  %%%%%%%%%%%%%%%%%%%%%%%%%%%%%%%%%%%%

% The abstract environment creates all the required headings and footers. 
% You only need to the text of the abstract in the file abstract.tex
\begin{abstract}
	% !TEX root = thesis_ruettgers_lukas.tex
% From mitthesis package
% Version: 1.01, 2023/06/19
% Documentation: https://ctan.org/pkg/mitthesis
%
% The abstract environment creates all the required headers and footnote. 
% You only need to add the text of the abstract itself.
%
% Approximately 500 words or less; try not to use formulas or special characters
% If you don't want an initial indentation, do \noindent at the start of the abstract

% TODO: Write Abstract
\noindent
The reasonable maximum demand on out-of-distribution generalization is that the learning algorithm should infer the \textit{simplest} function that remains consistent with the observed data.
%Out-of-distribution generalization requires agents to induce from their experience to unseen environments. 
%However, the reasonable maximum demand on inductive inference is to detect and infer the simplest underlying function that remains consistent with the observed data.
% TODO: Maybe replace information-theoretic by descriptive
Kolmogorov complexity intuitively formalizes the notion of simplicity from an information-theoretic viewpoint, as it associates the simplicity of a function with the description length of the shortest program that can compute it.
% Kolmogorov complexity intuitively formalises the descriptive complexity of a function, as it considers the length of the shortest program description that can produce it.
In light of this descriptive complexity, this thesis addresses limitations in our model classes, optimization techniques, and statistical learnability conditions that impede learning simple functions.
% In light of this descriptive complexity, current models and optimization techniques still lack the ability to learn recursive functions which are known to comprise simple yet powerful patterns.\\
% This thesis puts forward three arguments to illustrate these limitations.

Firstly, it demonstrates that models with a \textit{non-recursive structure} such as feed-forward neural networks are incapable of even expressing a class of functions that yet preserve a constantly low Kolmogorov complexity.
% To that end, it holistically considers what functions any such model could possibly express and defines a class of recursive functions that are inexpressible for these models yet preserve constantly low Kolmogorov complexity.
% To abstract from their specific architectures, it analyses this entire class of models from the viewpoint of first-order logical terms to consider what functions they could possibly express. 
% Upon this framework, it then defines a powerful class of recursive functions that yet preserve low Kolmogorov complexity.
% Then, it proposes a slight extension of first-order logic to recursive first-order logic and defines a class of decision problems over countably infinite alphabets upon it that capture powerful concepts and yet preserve low Kolmogorov complexity.
% And even if models were powerful enough to express such recursive patterns, ...
% This function class showcases another limitation that is independent of model expressivity and rather lies in the optimization formulations.
On the other extreme, models that allowed to express \textit{any partial computable function} would render learnability from finite datasets practically impossible with standard learning algorithms.
This is because present learnability conditions in theory either restrict themselves to exemplary model or hypothesis classes \cite{ahuja2021invariance,arjovsky2019invariant}, or rely on overly conservative assumptions such as the ubiquitous \textit{i.i.d.} assumption \cite{paccagnan2024pick,campi2023compression}.
% like ERM, IRM

In this setting of overarching model expressivity, this work instead substantiates how incorporating Kolmogorov complexity as a simplicity bias into the optimization objective function carves the way to formulate general distribution-free, both necessary and sufficient information-theoretic conditions that a dataset must satisfy to learn \textit{any} partial computable function, and exemplifies how such an inductive bias can further reduce the sample size that is usually required for classical learnability guarantees.
To that end, it proposes to directly draw upon Kolmogorov complexity to quantify the information that finite datasets convey about the functions that could have possibly generated it.
%, and illustrates how this quantity --- coined \textit{functional information} --- could realise simplicity biases in practice.

% that even dedicated domain generalization optimization objectives like Invariant Risk Minimization \cite{arjovsky2019invariant} still fail to identify the true hypothesis $h$ given finite datasets, because they do not favour simpler concepts over more complex ones.
% When however incorporating Kolmogorov complexity as a simplicity bias into the optimization objective, any dataset $D$ beyond a certain size will ensure that all non-recursive functions that still remain consistent with $D$ have a higher Kolmogorov complexity than $h$.
% Instead, it will assign the same optimal objective function value to infinitely many other hypotheses, because it does not favour simpler concepts.

Because Kolmogorov complexity is incomputable in general, compression algorithms are typically proposed as viable approximations \cite{cilibrasi2005clustering}. 
However, for \textit{any} choice on the encoding of Turing Machines, contemporarily employed compression algorithms such as the Lempel-Ziv-Welch compression can not even yield approximate guarantees about the order between the Kolmogorov complexity of two binary strings, because such compression algorithms do not keep up with the compression power of Turing Machines.
% TODO: Discrete parametrization of total computable functions is misleading.
% - (Prefix-free) encoding of Turing Machines $\not\equiv$ discrete parametrization of total computable functions
% TODO: Citation for the generalization improvement of regularization.
%Although regularization provably improves generalization within particular parametrized function classes, this thesis finally corroborates that for any choice on the encoding of Turing Machines, regularization can not even yield approximate guarantees about \textit{functional information}, because there are arbitrarily large order inconsistencies between the Kolmogorov complexity and the norm of binary vectors.
% For restricted parametrized function classes, regularization  is a widely used tool that controls the magnitude of the parameters and in this way provably improves generalization guarantees for functions realizable within the representation class.

% However, we show that regularization over any vector norm that would be practically feasible for regularization does not yield any simplicity guarantees for the function it encodes. 
% For arbitrarily large multiplicative and additive offset, there are bidirectional inconsistencies in the order that any such norm and Kolmogorov complexity induces on discrete-valued sequences.
% WORDING: compensation, offset (Ausgleich)

% The final argument concerns our theoretical approaches to yield generalization error bounds for hypothesis classes that are based on their complexity. 
% Finally, this work proves that if we regard functions by their Kolmogorov complexity instead of complexity measures employed in classical learning theory, it is not meaningful to condition learnability merely on the number of samples.
% Rademacher Complexity, VC dimension 
% Instead, it proposes to quantify such information-theoretic conditions in terms of Kolmogorov complexity too, and argues how it might empower generalization analysis beyond the omnipresent i.i.d. assumptions.




%%%%%%%%%%%%%%%%%% Alternative formulations %%%%%%%%%%%%%%%%%%%%%%%%%%%%%
% While recursion is a powerful tool for compressing algorithmic behaviour, the hardness of learning recursive descriptions has impeded its integration into modern machine learning pipelines.

% It emphasizes the efficacy of learning recursive algorithmic descriptions for reasonable behaviour outside of the training distribution, and showcases drawbacks in existing frequently employed models and optimization objectives in capturing elementary recursive patterns.
% use \input rather than \include because we're inside an environment
\end{abstract}

%%% acknowledgments.tex

% From mitthesis package
% Version: 1.01, 2023/10/16
% Documentation: https://ctan.org/pkg/mitthesis


\chapter*{Acknowledgments}
\addcontentsline{toc}{chapter}{Acknowledgments}

Write your acknowledgments here.
% .tex extension is presumed by \include 

%%% biography.tex
%% This section is optional

% From mitthesis package
% Version: 1.01, 2023/10/16
% Documentation: https://ctan.org/pkg/mitthesis

\chapter*{Biographical Sketch}
\addcontentsline{toc}{chapter}{Biographical Sketch}

Silas Whitcomb Holman was born in Harvard, Massachusetts on January 20, 1856. He received his S.B. degree in Physics from MIT in 1876, and then joined the MIT Department of Physics as an Assistant. He became Instructor in Physics in 1880, Assistant Professor in 1882, Associate Professor in 1885, and Full Professor in 1893. Throughout this period, he struggled with increasingly severe rheumatoid arthritis. At length, he was defeated, becoming Professor Emeritus in 1897 and dying on April 1, 1900.

Holman's light burned brilliantly before his tragic and untimely death. He published extensively in thermal physics, and authored textbooks on precision measurement, fundamental mechanics, and other subjects. He established the original Heat Measurements Laboratory. Holman was a much admired teacher among both his students and his colleagues. The reports of his department and of the Institute itself refer to him frequently in the 1880's and 1890's, in tones that gradually shift from the greatest respect to the deepest sympathy.

Holman was a student of Professor Edward C. Pickering, then head of the Physics department. Holman himself became second in command of Physics, under Professor Charles R. Cross, some years later. Among Holman's students, several went on to distinguish themselves, including: the astronomer George E. Hale ('90) who organized the Yerkes and Mt. Wilson observatories and who designed the 200 inch telescope on Mt. Palomar; Charles G. Abbot ('94), also an astrophysicist and later Secretary of the Smithsonian Institution; and George K. Burgess ('96), later Director of the Bureau of Standards. % optional, see MIT Libraries https://libraries.mit.edu/distinctive-collections/thesis-specs/#format


%%% Table of contents and lists of stuff (delete lists you don't need, e.g., if no tables) %%%%%%%%

\tableofcontents
\listoffigures
\listoftables


%%% Chapters of thesis  %%%%%%%%%%%%%%%%%%%%%%%%%%%%%%%%%%%%%%%%%%%%%%%%%%%%%%%%%%%%%%%%%%%%%%%%%%%

%% If you want to use "double spacing", you should start here...

 % !TEX root = thesis_ruettgers_lukas.tex
% From mitthesis package
% Version: 1.04, 2023/10/19
% Documentation: https://ctan.org/pkg/mitthesis

%% REFINEMENTS
% TODO: Alternatives for denote: denominate, designate
% TODO: Fix bib file. Wrap strings that should be capitalized by {}, e.g. {Mathematik}
% - In the context of supervised learning, we adopt the following terminology.

%% 2024/05/28: 6-7 pages of pure text
\chapter{Introduction} 
Think back to the moment where you and your classmates were taught elementary arithmetic on numbers.
Teachers across the world suggest their students to represent numbers in a particular numeral system, the decimal system.
In this vein, numbers are nothing but a sequence of symbols $a\in\{0,1,\dots,9\}$, and elementary arithmetic operations such as addition or multiplication can be briefly described as to inductively apply some simple symbol transformations over the symbol sequence.
Although the teacher gave you merely a few examples for small numbers, they contained sufficient information to describe this inductive algorithm, and enabled you to reproduce this algorithm for numbers beyond any witnessed scale.

%Now jump ahead to the moment when you were first introduced to the concept of prime numbers.
%You had already internalized the concept of divisibility by means of more fundamental concepts over the natural numbers.
The expedient lessons of this exemplary memory are twofold.
Firstly, it demonstrates that recursion is a simple, yet powerful mechanism to express unknown concepts with more fundamental ones and generalize these concepts to unknown problem instances.
Secondly, it signifies that this generalization ability of humans rests on some kind of inductive bias that favours inferring \textit{simpler} concepts over more difficult ones.
That way, the amount of information that the teacher needed to convey in terms of definitions and examples to let its students entirely comprehend the concept is related to some notion of the \textit{simplicity} of this concept.

%% RELATION OF COMPRESSION TO IN-DISTRIBUTION-GENERALIZATION
% - Benefit: far tighter, distribution-free bounds than classical statistical learning theory.
% - Caveat: Still i.i.d. assumption(!)

\section{Purpose and contributions}
In the setting of supervised learning, we also provide a learning algorithm examples in terms of functional mappings $(x,y)$ to eventually guarantee that it must detect the correct function that produces these mappings.

This thesis unravels how such a simplicity bias that draws on Kolmogorov complexity could enable the learnability of any computable function with finite information-theoretic resources, and even allows to undercut the sample sizes that are usually required to learn such functions in other learnability frameworks, such as the PAC model.
At the same time, it points up limitations in our models, learning algorithms, and learnability conditions that hinder us from learning simple functions.
Using the example of certain recursive functions, we\footnote{If the author uses the pronoun ``we'' in the subsequent work, he jointly refers to the readership and himself.} first demonstrate how feed-forward neural networks and any other model with a \textit{non-recursive structure} can not even express functions that are however simple to describe given what functions --- activation functions, arithmetic operations, constants, logical expressions, etc. --- they have access to.
These restraints of their representation class hence bias \textit{against} some simple functions.
On the other hand, learnability guarantees such as classical ones that are based on the Rademacher complexity or VC dimension of hypothesis classes rest on some limitations of the hypothesis class.
By quantifying the information in a dataset about the underlying function in terms of Kolmogorov complexity, such restraints however disappear, as the hypothesis class may now admit \textit{any} partial computable function and still render finite information sufficient to learn the true function.
Since Kolmogorov complexity $K$ is incomputable, this thesis finally touches on viable approximations of this quantity.
Although compression algorithms are frequently proposed in practice to estimate the simplicity within data, this thesis corroborates that any of the contemporary compression algorithms are unable to compress simple regularities in strings $v$ to an arbitrary scale, and therefore can not even yield approximate theoretical guarantees on the \textit{order} between the Kolmogorov complexity of two strings, $K(v)<K(w)$, even if we allow for any exponential, multiplicative, and additive approximation offsets.

\section{Content structure}
To begin with, Chapter \ref{chap:preliminaries} provides essential preliminaries and contextualizes our work in light of prior research. 
While Section \ref{sec:domain-generalization} presents pertinent work in the field of domain generalization and indicates remaining intricate problems, Section \ref{sec:kolmogorov-complexity} provides a comprehensive introduction to Kolmogorov complexity.

Subsequent to that, Chapter \ref{chap:models-optimization-kolmogorov} addresses the expressive limitations of non-recursive models and the discriminative flaws of optimization objectives.
To that end, Section \ref{sec:non-recursive-models} first rigorously defines \textit{non-recursive models}.
Then, Section \ref{sec:recursive-completion} introduces a class of recursive, simple functions that such models cannot express.
While the inexpressivity result is proven in Section \ref{sec:expressive-limits-non-recursive-models}, Section \ref{sec:uniform-simplicity-recursive-completion} shows that these functions have a constantly low Kolmogorov complexity given the functions these models have access to.
Finally, Section \ref{sec:optimization-objectives-lack-simplicity-bias} demonstrates that infinitely many functions that could be expressed by these non-recursive models achieve the optimal score according to standard optimization objectives like Empirical Risk Minimization or even Invariant Risk Minimization, because these models could essentially memorize the training data.
Thereon, it exemplifies how the discriminative capability of Kolmogorov complexity renders all these functions suboptimal for sufficient large finite datasets, and hence allows to learn the true recursive function.

Generalizing this insight, Chapter \ref{chap:sufficient-information-learnability} formulates sufficient information-theoretic conditions that allow to learn the overarching hypothesis class of computable functions, but still remain optimal, in the sense that less information would in general not allow to learn the respective function.
After Section \ref{sec:sample-size-does-not-avail-learnability} exemplifies that it is neither possible nor meaningful to merely condition learnability of computable functions on the number of samples in the dataset, Section \ref{sec:quantify-functional-information} alternatively elaborates on quantifying the information that datasets convey about the functions that could have generated it in terms of Kolmogorov complexity.
To that effect, it substantiates that weaker formulations of this information based on Kolmogorov complexity violate desirable properties, and thereby concludes with the ultimate formulation, coined \textit{functional information}.
Thereafter, it elucidates how this functional information smooths the way to the aforementioned learnability conditions, and juxtaposes them to prior works in learnability and compression. In particular, it exemplifies on the hypothesis class of $d$-dimensional parity functions that the additional discriminative power of the incorporated simplicity bias allows to learn parity functions with less than $d$ samples, given their Kolmogorov complexity is relatively small.
Finally, Section \ref{sec:functional-inforamtion-simplicity-bias} illustrates how an oracle for functional information could be used to realise a simplicity bias in practice.

Although Kolmogorov complexity is incomputable, a simplicity bias merely requires an approximation that could yield some approximate guarantee on the \textit{order} between the Kolmogorov complexity of two strings $K(v)<K(w)$.
However, Chapter \ref{chap:compression-kolmogorov-complexity} addresses why compression algorithms, which are frequently proposed as approximations of Kolmogorov complexity, do not even yield such approximate guarantees.
For this purpose, Section \ref{sec:limpel-ziv-welch-algorithm} uses the Lempel-Ziv-Welch algorithm as a classical example of compression algorithms to demonstrate their limitations in compressing strings that have been generated by simple algorithms.
After corroborating the arbitrary compression power of Turing Machines in Section \ref{sec:compressibility}, Section \ref{sec:unbounded-order-inconsistencies} instantiates that such compression algorithms can not yield such guarantees for any exponential, multiplicative, and additive approximation offsets.

Finally, we summarize the above results and their impact in Chapter \ref{chap:conclusion} and raise future research questions.
% .tex extension is presumed
 % !TEX root = thesis_ruettgers_lukas.tex
% From mitthesis package
% Version: 1.04, 2023/10/19
% Documentation: https://ctan.org/pkg/mitthesis


\chapter{Domain Generalization of Neural Networks}
\section{Extrapolation Drawbacks of Models}
\subsection{ReLU MLPs Extrapolate Linearly}
% \cite{hornik1989multilayer} MLPs are universal function approximators
% \cite{xu2019can} ReLU MLP extrapolation in NTK regime
% \cite{jacot2018neural} NTK
Within the support of the training distribution, MLPs are universal function approximators.
In the NTK regime, ReLU MLPs converge to linear functions outside the training distribution with a linear convergence rate.
Instead, the non-linearities in the architecture are the crucial foundation for encoding task-specific non-linearities. Compare Graph Neural Networks and Dynamic Programming Problems.
% TODO: Show own proof outside the NTK regime with regularized ReLU MLPs.

\section{Invariant Causal Mechanisms}
% TODO: At some point we need to introduce the notions of causality.
\section{Invariant Risk Minimization}
% TODO: Describe and criticize IRM.
% TODO: Understand the drawback of the linearization of IRM. At what cost did they derive IRMv1?

% \cite{arjovsky2019invariant} IRM
\subsection{Fully Informative Invariant Features}
% \cite{ahuja2021invariance} IB-IRM, FIIF vs. PIIF
But I argue that there is a far larger drawback.
\subsection{Learning Prime Numbers}
Consider the decision problem \textsc{PrimeNumbers}.
It is widely believed that there is no efficient algorithm that decides prime numbers.
% TODO: Choose citation for hardness of the prime number decision problem.

\section{Other Optimization Objective Reformulations}
\subsection{Risk Extrapolation}
%\cite{krueger2021out} REx
\subsection{Distributionally Robust Optimization}


Empirically, none of the above algorithms perform better than ERM.

 % !TEX root = thesis_ruettgers_lukas.tex
% From mitthesis package
% Version: 1.04, 2023/10/19
% Documentation: https://ctan.org/pkg/mitthesis


\chapter{Recursion and Simplicity}
% TODO: Read Minimum Description Length Literature
\section{Kolmogorov Complexity}
% Tentative title: Simplicity of Strings
% \cite{li2008kolmogorov}
% TODO: Cite initial paper of Kolmogorov.
%% CONTENT OVERVIEW
% - Conditional Kolmogorov Complexity
% - Invariance Theorem: Constant Factor for Changing Reference Machine
% - - Turing Machine, Goedelnumber ordering
% - - Criticism why this additive constant should matter in our case.
% - - - Prove that for any constant c, there is a reference machine which biases against the function f such that its Kolmogorov complexity must be larger than c
% - Explicit Description Upper Bound for Kolmogorov Complexity of Strings
% - Plain vs. prefix Kolmogorov complexity
% - Monotonic universal reference machine


% TODO: State that we assume the following universal Turing Machine.
% 1. It is the natural universal Turing Machine. The order of Turing Machines is determined by their Gödelnumber, which coincides with our inductive bias on computational simplicity.
% 2. Invariance Theorem. It only differs only by an additive constant. 
% TODO: Introduce Conditional Kolmogorov Complexity here, too.


%%%%%%%%%%% BASIC SETUP FOR KOLMOGOROV COMPLEXITY %%%%%%%%%%%%%%%%%%%%%%%%%%%%%%%%%%%%%

% TODO: Introduce G\"odel number
Denote by $\operatorname{enc}(\mathcal{T})$ the G\"odel number of a Turing Machine.

\begin{definition}[Kolmogorov Complexity]
	
\end{definition}


\subsection{The Implicit Bias within Kolmogorov Complexity}
\begin{theorem}[Invariance Theorem]
	There is only an additive constant between Kolmogorov Complexities of any pair of universal Turing Machines.
\end{theorem}

% TODO: Mention that this constant might become arbitrarily large. (At least later when we deal with Information Thresholds)
% TODO: This should be put as early as possible in the chapter.
In the following, we assume an arbitrary, fixed, finite alphabet $\Sigma$ of cardinality $|\Sigma|=:r$.
Without loss of generality, we identify $\Sigma$ with the field $\mathbb{Z}_r$ to which it possesses a natural isomorphism $\pi$.
When we speak of $0,1\in Sigma$, we refer to the elements $a,b\in\Sigma$ with $\pi(a)=0, \pi(b)=1$.
Since $r\in\mathbb{N}$ is finite, $\mathfrak{C}\cong \mathbb{Z}_r$ is equivalent to $\mathfrak{C}\equiv \mathbb{Z}_r$ for any mathematical structure $\mathfrak{C}$.
That is, $\mathfrak{C}$ will satisfy exactly those first-order logic formulas that $\mathbb{Z}_r$ satisfies.
In our case, this also holds for second-order logic formulas formulated over $\mathbb{Z}_r$, as quantification over predicates and relations over finite objects is expressible in first-order logic, too.
% TODO: Purely additive vector norms are only expressible in second-order logic.

\begin{definition}[Universal Turing Machine]
	\label{def:universaltm}
	We define a universal Turing Machine $U$ that receives both program index and input in a self-delimiting encoding.
	The construction is in the spirit of \cite[Section 2.1]{li2008kolmogorov} as follows. 
	% TODO: Check whether it's truly Section 2.1
	$U$ expects input in the form $1^n 0 xy$, where $x,y\in\{0,1\}^{*}$ and $l(x)=n$.
	$U$ hence interprets the number of leading $1$s as the length of the first input.
	It can uniquely identify the number of leading $1$s by the first appearance of $0$ on the input tape.
	$U$ interprets $x$ as the encoding of the index of the Turing Machine it shall execute and retrieves the Gödel number of this Turing Machine in the natural way.
	That is, $U$ incrementally iterates through all binary strings, checks whether they encode a Gödel number and subtracts $x$ by one each time it encounters a valid Gödel number.
	When $x$ has been reduced to the empty string, the current binary string will encode the $n_x$th valid Gödel number of some Turing Machine $\mathcal{T}_x$. 
	
	Then, $U$ simulates $\mathcal{T}_x$ on input $y$.
\end{definition}

\begin{definition}[Strings with Infinite Zero Padding]
	\label{def:zero-pad-string}
	For any $x\in\Sigma^{*}$, we define $x\overline{0}\in\Sigma^{\infty}$ as the infinite string with prefix $x$ and infinitely many subsequent $0$s.
\end{definition}

\begin{lemma}[Additional Complexity of Infinite Zero Padding]
	\label{lemma:additional-complexity-zero-pad}
	There is a constant $c_{pad}$ such that for every string $x\in\Sigma^{*}$, $K(x\overline{0})\leq K(x)+c_{pad}$.
\end{lemma}
\begin{proof}
	We construct a Turing Machine $\mathcal{T}_{pad}$ in the following way.
	$\mathcal{T}_{pad}$ always moves its tape head to the right. 
	It leaves symbols $a\in\Sigma$ unchanged and replaces every blank symbol $B$ by $0$.
	For any input $x$, $\mathcal{T}_{pad}(x)=x\overline{0}$.
	% TODO: How do we define the output of a Turing Machine that never halts but computes infinite sequences? Check \cite{li2008kolmogorov}.
	This Turing Machine is well-defined. In particular, it has a finite G\"odel number $\operatorname{enc}(\mathcal{T}_{pad})$ of length $c_{enc}\in\mathbb{N}$.
	There is an $p\in\mathbb{N}$ such that $\operatorname{enc}(\mathcal{T}_{pad})$ is the $p$th G\"odel number in the natural enumeration of G\"odel numbers that is also computed by our universal Turing Machine $U$ from Definition \ref{def:universaltm}. 
	Therefore, there exists a constant $c_0\leq\log_2(p)$ such that $K(\mathcal{T})=c_0$.
	
	As providing $x_p$ and $x$ to the universal Turing Machine $U$ in the self-delimiting format $1^{c_{0}} 0 x_p x$ yields $x\overline{0}$, we finally obtain $K(x\overline{0})\leq \underbrace{2c_0+1}{=:c_{pad}}+K(x)$.
\end{proof}

%
%
%
%
%
%
%
%
%%%%%%%%%%% BASIC PROPERTIES OF KOLMOGOROV COMPLEXITY %%%%%%%%%%%%%%%%%%%%%%%%%%%%%%%%%%
\subsection{Compressibility}
\begin{lemma}[Incompressible Strings]
	\label{lemma:incompressible-strings}
	For any $n\in\mathbb{N}$, there exists a string $v\in\Sigma^n$ with $K(v)\geq l(v)=n$.
	
\end{lemma}
\begin{proof}
	Let $r:=|\Sigma|$, thus $|\Sigma^n|=|\Sigma|^n=r^n$.
	All these strings are different objects and consequently must have different encodings.
	For any $v,w\in\Sigma^n$, if we have $v\neq w$ and $U(p)=v$ and $U(q)=w$ for some universal reference machine $U$ and programs $p,q\in\Sigma^*$, then necessarily $p\neq q$.
	But the geometric sum $\sum_{i=0}^{n-1}r^i=r^{n}-1$, hence there are only $r^n-1$ strings in $\Sigma^{*}$ with length at most $n-1$.
	By the pigeonhole principle, there must be at least one $v\in\Sigma^n$ such that there exists no program $p$ of length $l(p)<n$ with $U(p)=v$.
	For this $v$, we have $K(v)\geq n=l(v)$.
\end{proof}
We extend the above lemma to infinite strings with finite hamming weight.
\begin{corollary}[Incompressible Strings with Infinite Paddings]
	\label{cor:incompressible-zero-pad}
	For any $n\in\mathbb{N}$, there exists a string $v_n=x_n\overline{0}\in\Sigma^{\infty}$ with $K(v_n)\geq l(x_n)=n$.
\end{corollary}
\begin{proof}
	The argument is analogous to Lemma \ref{lemma:incompressible-strings}.
	For any $n\in\mathbb{N}$, there are exactly $2^n$ strings $v\in\Sigma^{\infty}$ that satisfy $v=x\overline{0}$ for some $x\in\Sigma^{*}$. However, there are only $2^n-1$ strings of length shorter than $n$ that could serve as encodings for strings like $v$.
	Therefore one of the $2^n$ strings is not compressible beyond length $n$.
\end{proof}

% Constructive Logarithmic Complexity Upper Bound
\begin{lemma}[Logarithmically Compressible Strings]
	\label{lemma:logcompress}
	There exists a constant $c\in\mathbb{N}$ such that:
	For every $n\in\mathbb{N}$, there exists a string $z_n\in\Sigma^{*}$ of length $l(z_n)=n$ with Kolmogorov complexity $K(z_n)\leq \log (n) + c$.
\end{lemma}
\begin{proof}
	Without loss of generality, $0,1\in\Sigma$. Since $\Sigma$ is finite, it is isomorphic $\{0,1,\dots,r-1\}$, where $r=|\Sigma|$.
	% TODO: State this somewhere in the beginning of this section. We will use it multiple times.
	We construct a Turing Machine $\mathcal{T}$ as follows:
	If the string $x$ on the input tape does not only consist of $0$s and $1$s, $\mathcal{T}$ immediately terminates and hence outputs $\varepsilon$.
	In the following, we assume $x\in\{0,1\}^{*}$.
	$\mathcal{T}$ interprets $x$ as the encoding of a natural number in the usual way. Leading zeros are ignored and immediately replaced by the blank symbol $B$.
	We denote $n_x\in\mathbb{N}$ as the natural number encoded by $x$. 
	Conversely, for any $n\in\mathbb{N}$, $x_n$ denotes the unique binary string encoding it without leading zeros. Accordingly, $0$ is encoded by the empty string $\varepsilon$.
	$\mathcal{T}$ now repeats the following procedure until the input tape only holds the empty string $\varepsilon$.
	\begin{itemize}
		\item Write a $1$ on the next free box.
		\item Subtract the input by $1$.
	\end{itemize}
	Finally, $\mathcal{T}$ returns a string of $n_x$ subsequent $1$s.
	In particular, the partial computable function $f$ computed by $\mathcal{T}$ satisfies $f(n_x)=1^{n}$ for each $n\in\mathbb{N}$.
	Analogously as in the proof of Lemma \ref{lemma:additional-complexity-zero-pad}, there exists a string $x_p$ and a constant $c_\mathcal{T}\leq\log_2(p),p\in\mathbb{N}$, such that $K(\mathcal{T})=c_\mathcal{T}$. 
	
	For any $n\in\mathbb{N}$, providing $1^{c_\mathcal{T}} 0 x_p x_n$ to the universal Turing Machine $U$ from Definition \ref{def:universaltm} as input yields $z_n:=1^n$ as output.
	For that reason, $K(z_n)\leq 2c_\mathcal{T} + 1 + l(x_n) \leq \underbrace{2c_\mathcal{T}+2}_{=:c}+\log(n)$.
	
\end{proof}

\begin{corollary}[Log-Compressible Strings with Infinite Zero Padding]
	\label{cor:log-compressible-zero-pad}
	There exists a constant $c\in\mathbb{N}$ such that:
	For every $n\in\mathbb{N}$, there exists a string $p_n\in\Sigma^{\infty}$ such that $p_n=z_n\overline{0}$ for a string $z_n$ of length $l(z_n)=n$ and that has Kolmogorov complexity $K(p_n)\leq \log (n) + c$.
\end{corollary}
\begin{proof}
	By virtue of Lemma \ref{lemma:additional-complexity-zero-pad}, padding the strings of the form $1^{n}$ with infinite $0$s required only a constant additional amount $c_{pad}$ of descriptive information.
	Combined with the constant $c$ from the proof of Lemma \ref{lemma:logcompress}, we obtain $K(p_n)\leq \log(n) + c + c_{pad}$ for $p_n=1^n\overline{0}$.
\end{proof}

% Constructive Arbitary Complexity Upper Bound
\begin{lemma}[Arbitrarily Compressible Strings]
	\label{lemma:arbitrarycompress}
	There exists a constant $c\in\mathbb{N}$ such that:
	For every $n\in\mathbb{N}$, there exists a string $z_n\in\Sigma$ of length $l(z_n)=\left(\bigcirc_{i=1}^n \exp_2\right)(1)$ with Kolmogorov complexity $K(z_n)\leq \log (n) + c$.
\end{lemma}
\begin{proof}
	This time, we construct a Turing Machine $\mathcal{T}$ that computes an exponential tower function. 
	The function $f$ computed by $\mathcal{T}$ will satisfy $f(x_n)=\begin{cases}
		1^{\left(\bigcirc_{i=1}^n \exp_2\right)(1)}, & n\in\mathbb{N},n\geq 1\\
		1, & n=0\in\mathbb{N}.\\
	\end{cases}$
	
	This is done in the following way:
	Just as in the proof of Lemma \ref{lemma:logcompress}, $\mathcal{T}$ interprets the input $x$ as the encoding of a natural number $n_x$.
	Besides the input tape and the output tape, $\mathcal{T}$ uses one auxiliary tape. 
	If $x$ does not satisfy the expected format, $\mathcal{T}$ immediately terminates.
	Otherwise, $\mathcal{T}$ first writes a $1$ on its output tape.
	Now, it repeats the following procedure while the string on the input tape is not eradicated blank.
	% TODO: Find a better wording than "eradicating blank".
	\begin{itemize}
		\item Remove all non-blank symbols on the auxiliary tape.
		\item Copy the non-blank symbol sequence $x$ on the output tape to the auxiliary tape.
		\item Interpret $x$ as the natural number $n_x\in\mathbb{N}$. Compute $x_{2^{n_x}}$ just as in the proof for Lemma \ref{lemma:logcompress} and write it on the output tape.
		\item Subtract the number on the input tape by $1$.
	\end{itemize}
	
	The string $x$ on the output tape now represents $n_x=\left(\bigcirc_{i=1}^n \exp_2\right)(1)$.
	Finally, copy $x$ once more to the auxiliary tape and clean the output tape. 
	Now, repeat the following procedure as long as the string on the auxiliary tape is not eradicated blank.
	\begin{itemize}
		\item Write a $1$ on the output tape.
		\item Subtract the string on the auxiliary tape by $1$.
	\end{itemize}
	
	With $1^{\left(\bigcirc_{i=1}^n \exp_2\right)(1)}$ written to the output tape, $\mathcal{T}$ terminates.
	
	By the same argument as in the proof of Lemma \ref{lemma:logcompress}, there is a program string $p$ that identifies $\mathcal{T}$ with a length of $c_{\mathcal{T}}\in\mathbb{N}$.
	For any $n$, providing $1^{c_\mathcal{T}} 0 p x_n$ to the universal Turing Machine $U$ yields the output $z_n:=1^{\left(\bigcirc_{i=1}^n \exp_2\right)(1)}$ of length $\left(\bigcirc_{i=1}^n \exp_2\right)(1)$.
	Henceforth, $K(z_n)\leq \log(n) + c$ for $c:=2c_\mathcal{T}+2$.
	
\end{proof}
\begin{corollary}[Arbitrarily Compressible Strings with Infinite Zero Padding]
	\label{cor:arbitrarycompress-zero-pad}
	There exists a constant $c\in\mathbb{N}$ such that:
	For every $n\in\mathbb{N}$, there exists a string $p_n\in\Sigma^{\infty}$ such that $p_n=z_n\overline{0}$ for a string $z_n$ of length $l(z_n)=\left(\bigcirc_{i=1}^n \exp_2\right)(1)$ and that has Kolmogorov complexity $K(p_n)\leq \log (n) + c$.
\end{corollary}
\begin{proof}
	The proof works just as in Corollary \ref{cor:log-compressible-zero-pad} and merely requires replacing $c$ and $p_n$ by the respective values in the proof of Lemma \ref{lemma:arbitrarycompress}.
\end{proof}

\section{Regularization and Kolmogorov Complexity}

% TODO: Somewhere, we need to introduce the big concatenation operator.
% TODO: We need to introduce that we mean \exp_2(\cdot)\equiv 2^{I(\cdot)}, where $I$ is the identity function.
%
%
%
%
%
%
%
%
%%%%%%%%%%% INVARIANCE KOLMOGOROV COMPLEXITY UNDER PERMUTATION, WHILE REGULARIZATION IS NOT %%%%%%%%%%%%%%%%%%%%%%%%%%%%%%%%%%
% Is a sequence 90909090909090... more complex than 101001000010000000010000....?
% By the homogeneity of vector norms, regularization biases towards symbols with a smaller ordinal position.
% But Kolmogorov Complexity is invariant under permutation.

\begin{definition}[Turing Machine for Permutation]
	\label{def:permutationtm}
	Let $\pi$ be an arbitrary permutation over $\Sigma$.
	We define $\mathcal{T}_\pi$ in the straightforward way.
	$\mathcal{T}$ goes over the input and replaces every symbol $a\in\Sigma$ by $\pi(a)$. 
	After having arrived on the right end, it moves back to the first symbol and terminates.
\end{definition}
\begin{table}[h]
\begin{center}
\begin{tabular}{| c | c | c c c |c |}
\toprule
$\delta$ & $0$ & $1$ & \dots & $r-1$ & $B$ \\\hline
$q_0$ & $(q_0,\pi(0),R)$ & $(q_0,\pi(1),R)$ & \dots & $(q_0,\pi(r-1),R)$ & $(q_2,B,L)$ \\\hline
$q_2$ & $(q_2,0,L)$ & $(q_2,1,L)$ & \dots & $(q_2,r-1,L)$ & $(q_1,B,R)$ \\\bottomrule
\end{tabular}
\end{center}
\caption[Transition function of permutation Turing Machine]{The schematic transition function $\delta$ of $\mathcal{T}_\pi$. By convention, $q_1$ represents the final state and is hence omitted.}
\label{tab:tmpermutation}
\end{table}

% First put forward a weak argument (Permutation, Binary Inverse) that illustrates the disparity between Kolmogorov complexity and vector norms
It is trivial to show that the length of the G\"odel number of $\mathcal{T}_\pi$ is the same for all permutations $\pi$ over $\Sigma$, namely $c:=l(\mathcal{T}_\pi)=a\cdot |\Sigma|+b$ for small constants $a,b\in\mathbb{N}$. 
The interested reader is referred to Lemma \ref{lemma:permutation-complexity-bound} in the Appendix.

As is rigorously proved thereafter in Lemma \ref{lemma:optimality-permutation-tm}, the Turing Machine $\mathcal{T}_\pi$ from Definition \ref{def:permutationtm} is optimal in terms of its G\"odel number length and therefore also in terms of the Kolmogorov complexity of $\pi$.
For any permutation $\pi$ except the identity function, there is no Turing Machine $\mathcal{T}$ that computes $\pi$ and achieves G\"odel number of shorter length than $\mathcal{T}_\pi$.
Moreover, this optimal length is invariant across all permutations over $\Sigma$ - except the identity function.
With Lemma \ref{lemma:permutation-complexity-bound}, the interested reader will also find a tight bound for the length of $\operatorname{enc}(\mathcal{T})_\pi$.

\begin{theorem}[Kolmogorov Complexity Invariance under Permutation]
	There exists a small constant $c_{\Sigma}$ that scales only quadratically with $|\Sigma|$ such that the following holds:
	Let $\pi$ be an arbitrary permutation over $\Sigma$ that is not equivalent to the identity function. 
	Let $x\in\Sigma^{*}$ be an arbitrary string over $\Sigma$.
	Then $|K(x)-K(\pi(x))|\leq c_{\Sigma}$.
\end{theorem}
\begin{proof}
	Since $\pi$ is a permutation, it has an inverse function $\pi^{-1}$.
	Since $\pi$ is not equivalent to the identity function, $\pi^{-1}$ is neither.
	By Lemma \ref{lemma:permutation-complexity-bound}, $K(\pi),K(\pi^{-1})\leq c_{0}:=2|\Sigma|^2+31|\Sigma|+34$.
	
	Let $x\in\Sigma^{*}$ be arbitrary. Denote by $\pi(x)\in\Sigma^{*}$ the string that is obtained by applying $\pi$ to every symbol in $x$.
	
	Then, providing $z=1^{c_{0}}0 \operatorname{enc}(\mathcal{T}_\pi) x$ to the universal Turing Machine $U$ from Definition \ref{def:universaltm} yields $U(z)=\pi(x)$.
	Consequently, $K(\pi(x))\leq 2\cdot c_0+1+K(x)$.
	
	Since the above argument holds for any string $x\in\Sigma^{*}$, it also holds for $\pi^{-1}(x)$ for any $x\in\Sigma^{*}$.
	Therefore, we symmetrically have $K(x)\leq 2\cdot c_0+1 K(\pi^{-1}(x))$ and hence $K(x)\leq 2\cdot c_0+1 K(\pi(x))$.
	Stitching these two bounds together and choosing $c_\Sigma:=2c_0+1$, we finally obtain the desired result.
\end{proof}

%
%
%
%
%
%
%
%
%%%%%%%%%%% UNBOUNDED INCONSISTENCIES BETWEEN KOLMOGOROV COMPLEXITY AND VECTOR NORMS %%%%%%%%%%%%%%%%%%%%%%%%%%%%%%%%%%
% People might wonder whether there are at least some additive and multiplicative constants under which the order is preserved.
% We prove that this is not the case.


By elementary calculus, the following result is established. The proof is omitted here and laid out in Appendix \ref{sec:elementary-math} instead.
\begin{lemma}[Log-Linear Inequality with Additive Constant]
	\label{lemma:log-lin-add-inequality-placeholder}
	Let $a\geq 1,b\geq 0$ be arbitrary real numbers.
	For any real number $x > 2^{4(a+b)}$, it holds that $a\log_2(x)+b < x$.
\end{lemma}

\begin{definition}[Properly additive vector norms]
	\label{def:properly-additive-vector-norms}
	Let $\lVert \cdot \rVert:\Sigma^{\infty}\to\mathbb{R}_{\geq 0}$ be an arbitrary norm over $\Sigma^{\infty}$.
	% TODO: Replace $\Sigma$ by $\mathbb{Z}_r$?
	Denote by $e_1,e_2,\dots$ the unit vectors that form an orthogonal basis of $\Sigma^{\infty}$. That is, $e_i$ consists only out of $0s$ except for index $i$, where it carries a $1$.

	We say that $\lVert \cdot \rVert$ is \textit{properly additive} if for any proper subset $A\subset \mathbb{N}$ and $v_A:=\sum_{i\in A}e_i$, and any $j\notin A$, it holds that
	\begin{equation}
		\lVert v_A + e_j \rVert > \lVert v_A \rVert.
	\end{equation}
	
	Vector norms that do not satisfy this condition are called \textit{improperly additive}.
\end{definition}
% TODO: What about the properties of improperly additive vector norms?

With this lemma an definition at hand, we now show the following.
\begin{theorem}[Unbounded Order Inconsistencies to any Vector Norm]
	Let $\lVert\cdot\rVert$ denote an arbitrary properly additive norm over $\Sigma^{\infty}$ as by Definition \ref{def:properly-additive-vector-norms}.
	% TODO: We need to formally define why $\Sigma^{\infty}$ identifies a vector space.
	For any pair of real numbers $a\geq 1, b\geq 0$, there are $v,w\in\Sigma^{\infty}$ such that $K(v)\geq a\cdot K(w)+b$, but $\lVert v \rVert < \lVert w \rVert $. 
\end{theorem}
\begin{proof}
	Let $a\geq 1, b\geq 0$ be arbitrary real numbers.
	Let $c\in\mathbb{N}$ be the constant from the proof of Corollary \ref{cor:log-compressible-zero-pad}.
	For any $n\geq ac+b$, define $f_{a,b}$ as $f_{a,b}(n):=2^{\lfloor \frac{n}{a}-c-\frac{b}{a}\rfloor}$.
	Fix an arbitrary $n> \max(ac+b,2^{a(c+2)+b})$.
	
	By Corollary \ref{cor:incompressible-zero-pad}, there must exists a string $q_n\in\Sigma^{\infty}$ with $q_n=r_n\overline{0}$ for some string $r_n\in\Sigma^n$ such that $K(q_n)\geq n$. 
	
	Now, consider the string $p_{n}=z_{f_{a,b}(n)}\overline{0}$ from the proof of Corollary \ref{cor:log-compressible-zero-pad}.
	By definition, $z_{f_{a,b}(n)}=1^{f_{a,b}(n)}$.
	Moreover, it holds that $K(p_{n})\leq \log_2(f_{a,b}(n))+c$.
	But by definition, $f_{a,b}(n)= 2^{\lfloor \frac{n}{a}-c-\frac{b}{a}\rfloor}\leq 2^{\frac{n}{a}-c-\frac{b}{a}}$.
	
	Therefore, $K(p_n)\leq \frac{n}{a}-c-\frac{b}{a}+c=\frac{n}{a}-\frac{b}{a}$.
	In total, we have 
	\begin{equation}
		K(q_n)\geq n = a\left(\frac{n}{a}-\frac{b}{a}\right)+b\geq a\cdot K(p_n)+b.
	\end{equation}
	
	At the same time, the non-zero prefix $z_{f_{a,b}(n)}$ of $p_n$ is longer than the non-zero prefix $r_n$ of $q_n$.
	By definition, $z_{f_{a,b}(n)}$ has length $2^{\lfloor \frac{n}{a}-\frac{b}{a}-c\rfloor}\geq 2^{ \frac{n}{a}-\frac{b}{a}-c-1}$.
	But as $n>\max(2^{a(c+2)+b},2)$, Lemma $\ref{lemma:log-lin-add-inequality-placeholder}$ yields
	$n\geq a\log_2(n)+a(c+1)+b$. By the monotonicity of the exponential function $\exp$, this implies
	\begin{equation}
		2^{n}\geq 2^{a\log_2(n)+a(c+1)+b}=n^a \cdot 2^{a(c+1)+b}.
	\end{equation}
	
	Latching onto this line of reasoning, the monotonicity of the square root function $\sqrt[a]{\cdot}$ now yields
	\begin{equation}
	2^{\frac{n}{a}}=\sqrt[a]{2^{n}}\geq \sqrt[a]{ n^a \cdot 2^{a(c+1)+b}} =n \cdot 2^{c+1+\frac{b}{a}}.
	\end{equation}
	
	Finally, dividing by the fixed factor $2^{c+1+\frac{b}{a}}$ culminates in
	\begin{equation}
		\label{eq:order-inconsistency-to-norm-length-inequality}
		l\left(z_{f_{a,b}(n)}\right)=l\left(1^{f_{a,b}(n)}\right)=f_{a,b}(n)\geq 2^{ \frac{n}{a}- \left(\frac{b}{a}+c+1\right)}\geq n = l(r_n).
	\end{equation}
	
	Since $z_{f_{a,b}(n)}$ consists entirely out of $1$s and is longer than $r_n$, there must exist a $w\in\{0,1\}^{*}$ such that $z_{f_{a,b}(n)}=r(n)+w$, and hence $p_n=q_n+w\overline{0}$, where addition is element-wisely identified as over the field $\mathbb{Z}_{r}, r:=|\Sigma|$.
	% TODO: Maybe define addition already a bit earlier.
	
	By Equation \ref{eq:order-inconsistency-to-norm-length-inequality} and the definition of $w$, there exist proper subsets $A,B,C\subset \mathbb{N}$ such that $q_n = \sum_{i\in A}e_i, w\overline{0}=\sum_{i\in B}e_i,$ and $p_n = \sum_{i\in C}e_i$.
	These sets satisfy $A\cap B = \emptyset, A\cup B = C$.
	Since $q_n\neq p_n$, we further have $B\neq \emptyset$.
	
	Since $\lVert\cdot\rVert$ is properly additive, repeated application of Definition \ref{def:properly-additive-vector-norms} yields
	\begin{equation}
		\lVert p_n \rVert = \left\lVert \sum_{i\in C}e_i \right\rVert \overset{A\subset C}{>} \left\lVert \sum_{i\in A}e_i \right\rVert = \lVert q_n \rVert,
		% TODO: This does not hold in general.
	\end{equation}
	which concludes our proof.
	
\end{proof}

% TODO: Argue that improperly additive norms make little sense of regularization
% - Maximum norm as an edge case example.

% TODO: There is no need for the following result, possibly remove it:
% - Padding Incompressible Strings

% But what can we derive if the norm of some vector v is m times smaller than the norm of another vector w?
\begin{lemma}[Padding Incompressible Strings]
	
\end{lemma}
\begin{proof}
	% Take a sufficiently long incompressible string.
	% Define a program that computes a bijection between strings and strings with hamming weight ratio at most p, for some rational 0<p<1/2.
	% The Kolmogorov complexity of these strings are bounded by the incompressible input and some constant. 
	% Otherwise they could be used to retrieve the incompressible input, because the Turing Machine computes a (computable) bijection.
	% But then the Kolmogorov Complexity of the original incompressible input string would be smaller, and this leads to a contradiction.
\end{proof}

\begin{definition}[Super-logarithmic vector norms]
	\label{def:super-log-norm}
	Let $\lVert \cdot \rVert:\Sigma^{\infty}\to\mathbb{R}_{\geq 0}$ be an arbitrary norm over $\Sigma^{\infty}$.
	Denote by $e_1,e_2,\dots$ the unit vectors that form an orthogonal basis of $\Sigma^{\infty}$. That is, $e_i$ consists only out of $0s$ except for index $i$, where it carries a $1$.
	
	
	We say that $\lVert \cdot \rVert$ is \textit{super-logarithmic} if there exist constants $c_{\lVert\cdot\rVert},k_{\lVert\cdot\rVert}\in\mathbb{N}$ and $m_0\in\mathbb{N}$ such that 
	\begin{equation}
		\left\lVert \sum_{i=1}^{m}e_i \right\rVert\geq c_{\lVert\cdot\rVert}\left(\bigcirc_{i=1}^{k_{\lVert\cdot\rVert}}\log_2\right)(m) \quad \text{ for all } m\geq m_0.  
	\end{equation}
\end{definition}

\begin{lemma}[Basis Cascade Exponential Equality]
	\label{lemma:cascade-exp-inequality-basis-placeholder}
	Let $a\geq 1,b\geq 0$ be arbitrary real numbers.
	For any $k\in\mathbb{N}$ and $n\in\mathbb{N}$ with $n>4\cdot 2^k\cdot (a+b)$, it holds that
	\begin{equation}
		\left(\bigcirc_{i=1}^{n-k}\exp_2\right)(1) > a\cdot n + b.
	\end{equation}
\end{lemma}
\begin{theorem}[Unbounded Order Inconsistencies from any Super-Logarithmic Vector Norm]
	Let $\lVert \cdot \rVert$ denote an arbitrary norm that is super-logarithmic as by Definition \ref{def:super-log-norm}.
	For any real numbers $a\geq 1, b\geq 0$, there are $v,w\in\Sigma^{\infty}$ such that $\lVert v \rVert\geq a\cdot \lVert w \rVert+b$, but $K(v) < K(w)$. 
\end{theorem}
\begin{proof}
	Let $\lVert \cdot \rVert$ be an arbitrary norm that is super-logarithmic with some constants $c_{\lVert\cdot\rVert},k_{\lVert\cdot\rVert}$.
	Let $a\geq 1, b\geq 0$ be arbitrary real numbers.
	
	Denote by $c_{\mathcal{T}}\in\mathbb{N}$ the constant from the proof of Corollary \ref{cor:arbitrarycompress-zero-pad}.
	
	For any $n\in\mathbb{N}$, Corollary \ref{cor:incompressible-zero-pad} guarantees the existence of a string $q_n\in\Sigma^{\infty}$ with $q_n=r_n\overline{0}$ for some string $r_n\in\Sigma^n$ such that $K(q_n)\geq n$. 
	Let $g:\mathbb{N} \to \Sigma^{\infty}$ be any selection function that maps $n\in\mathbb{N}$ to such a string $q_n$. In the following, we therefore denote $q_n=g(n)$.
	
	On the other hand, define $p_n:=1^{\left(\bigcirc_{i=1}^n \exp_2\right)(1)}\overline{0}$ for $n\in\mathbb{N}$.
	
	Finally, we define $f_{a,b}(n):=\left(\bigcirc_{i=1}^{k_{\lVert\cdot\rVert}} \exp_2\right)\left(\frac{a\cdot \lVert q_n \rVert+b}{c_{\lVert\cdot\rVert}}\right)$ for any $n\in\mathbb{N}$.
	
	Now, assume that $n>\max \left(4\cdot 2^{k_{\lVert\cdot\rVert}}\cdot \frac{a+b}{c_{\lVert\cdot\rVert}}, 2^{4\cdot (1+c_{\mathcal{T}})}\right)$.
	
	For the first part, we use $n>4\cdot 2^{k_{\lVert\cdot\rVert}}\cdot \frac{a+b}{c_{\lVert\cdot\rVert}}$ and combine Lemma \ref{lemma:cascade-exp-inequality-basis-placeholder} with the monotonicity of the exponential function $\exp_2$ to obtain
	\begin{align}
		\label{eq:exp-f_ab-inequality-1}
		\left(\bigcirc_{i=1}^n \exp_2\right)(1) & = \left(\bigcirc_{i=1}^{k_{\lVert\cdot\rVert}} \exp_2\right)\left( \left(\bigcirc_{i=1}^{n-k_{\lVert\cdot\rVert}} \exp_2\right)(1) \right)\\
		& > \left(\bigcirc_{i=1}^{k_{\lVert\cdot\rVert}} \exp_2\right)\left( \frac{a\cdot n + b }{c_{\lVert\cdot\rVert}}\right).
	\end{align}
	
	Since $\lVert\cdot\rVert$ is a norm, it satisfies the triangle inequality.
	For any $n\in\mathbb{N}$, denote by $A_n\subset \mathbb{N}$ the set such that $q_n=\sum_{i\in A_n}e_i$.
	As $\lVert e_i \rVert = 1$ by definition of $e_i$, we have 
	\begin{equation}
		\label{eq:q_n-triangle-basis-vector-inequality}
		n\geq \sum_{i\in A_n}\lVert e_i \rVert \geq\left\lVert \sum_{i\in A_n} e_i\right\rVert = \lVert q_n \rVert.
	\end{equation}
	
	By the monotonicity of the exponential function $\exp_2$, we thereby culminate in
	\begin{align}
		\label{eq:exp-f_ab-inequality-full}
		\left(\bigcirc_{i=1}^n \exp_2\right)(1) &\overset{\ref{eq:exp-f_ab-inequality-1}}{>} \left(\bigcirc_{i=1}^{k_{\lVert\cdot\rVert}} \exp_2\right)\left( \frac{a\cdot n + b }{c_{\lVert\cdot\rVert}}\right) \\
		&\overset{\ref{eq:q_n-triangle-basis-vector-inequality}}{\geq} \left(\bigcirc_{i=1}^{k_{\lVert\cdot\rVert}} \exp_2\right)\left( \frac{a\cdot  \lVert q_n \rVert + b }{c_{\lVert\cdot\rVert}}\right)=f_{a,b}(n).
	\end{align}
	
	Along with Definition \ref{def:super-log-norm}, our condition is therefore satisfied:
	\begin{align}
		\lVert p_n\rVert = \left\lVert 1^{\left(\bigcirc_{i=1}^n \exp_2\right)(1)}\overline{0}\right\rVert 
		& =  \left\lVert \sum_{i=1}^{\left(\bigcirc_{i=1}^n \exp_2\right)(1)}e_i \right\rVert \\
		&\overset{\ref{def:super-log-norm}}{\geq} c_{\lVert\cdot\rVert}\left(\bigcirc_{i=1}^{k_{\lVert\cdot\rVert}}\log_2\right) \left( \left( \bigcirc_{i=1}^n \exp_2 \right)(1)\right) \\ 
		&\overset{\ref{eq:exp-f_ab-inequality-full}}{\geq} c_{\lVert\cdot\rVert}\left(\bigcirc_{i=1}^{k_{\lVert\cdot\rVert}}\log_2\right) \left( f_{a,b}(n) \right) \\ 
		&= c_{\lVert\cdot\rVert}\left(\bigcirc_{i=1}^{k_{\lVert\cdot\rVert}}\log_2\right) \left( \left(\bigcirc_{i=1}^{k_{\lVert\cdot\rVert}} \exp_2\right) \left(\frac{a\cdot \lVert q_n \rVert+b}{c_{\lVert\cdot\rVert}}\right) \right) \\
		&= c_{\lVert\cdot\rVert} \left(\frac{a\cdot \lVert q_n \rVert+b}{c_{\lVert\cdot\rVert}}\right)\\
		&= a\cdot \lVert q_n \rVert+b.
	\end{align}
	
	However, since $n> 2^{4(1+c_{\mathcal{T}})}$ and by the assumptions on the compressibility of $p_n$ and $q_n$, we conclude with Corollaries \ref{cor:arbitrarycompress-zero-pad} and \ref{cor:incompressible-zero-pad} that
	\begin{align}
		K(p_n)\overset{\ref{cor:arbitrarycompress-zero-pad}}{\leq}\log_2(n)+c_{\mathcal{T}} \overset{\ref{lemma:log-lin-add-inequality-placeholder}}{<} n \overset{\ref{cor:incompressible-zero-pad}}{\leq} K(q_n).
	\end{align}
\end{proof}

\begin{corollary}[Regularization with Minkowski Norms yields no Simplicity Guarantees]
	Let $p>0$ be an arbitrary real number and $\lVert\cdot\rVert_p$ denote the corresponding Minkowski norm.
	For any real numbers $a\geq 1, b\geq 0$, there are both $v,w\in\Sigma^{\infty}$ such that $K(v)\geq a\cdot K(w)+b$, but $\lVert v \rVert_p < \lVert w \rVert_p $, and $v,w\in\Sigma^{\infty}$ such that $\lVert v \rVert_p \geq a\cdot \lVert w \rVert_p+b$, but $K(v) < K(w)$. 
\end{corollary}
\begin{proof}
	Let $p>0$ be an arbitrary real number.
	
	We show that the Minkowski norm $\lVert\cdot\rVert_p$ is both properly additive (Definition \ref{def:properly-additive-vector-norms}) and super-logarithmic (Definition \ref{def:super-log-norm}).
	
	For the first part, let $A\subset \mathbb{N}$ be arbitrary.
	For any $j\in\mathbb{N}$ with $j\notin A$, the monotonicity of the root function $\sqrt[p]{\cdot}$ ensures that
	\begin{align}
		\left\lVert \sum_{i\in A} e_i + e_j \right\rVert_p &= \left(\sum_{i\in A\cup\{j\}} 1^p \right)^{\frac{1}{p}} = \left( |A|+1 \right)^{\frac{1}{p}} > |A|^{\frac{1}{p}} = \left(\sum_{i\in A} 1^p \right)^{\frac{1}{p}} = \left\lVert \sum_{i\in A} e_i \right\rVert_p.
	\end{align}
	
	For the second part, take $c=1,k=1,$ and choose a natural number $m_0>\max\left(\left(2^{4p}\right)^p,2^{4p}\right)$.
	
	Any $m\geq m_0$ in particular satisfies $m > 2^{4p}$.
	Therefore, Lemma $\ref{lemma:log-lin-inequality}$ guarantees that $m > p\log(m)$ for such $m$.
	
	There exist a real number $x$ such that $m=x^p$.
	Since $m>\left(2^{4p}\right)^p$, this $k$ satisfies $k>2^{4p}$. 
	For that reason, it holds for any $m\geq m_0$ that
	\begin{align}
		\left\lVert \sum_{i=1}^{m} e_i \right\rVert_p = \sqrt[p]{m} = \sqrt[p]{k^p} = k > p \log_2(k) = p\cdot \log_2(m^{\frac{1}{p}}) = p \cdot \frac{1}{p} \log_2(m)=\log_2(m).
	\end{align}
\end{proof}

\begin{corollary}[L0-Regularization yields no Simplicity Guarantees]
	Denote by $\lVert\cdot\rVert_0$ the norm that counts non-zero components.
	For any real numbers $a\geq 1, b\geq 0$, there are both $v,w\in\Sigma^{\infty}$ such that $K(v)\geq a\cdot K(w)+b$, but $\lVert v \rVert_0 < \lVert w \rVert_0 $, and $v,w\in\Sigma^{\infty}$ such that $\lVert v \rVert_0 \geq a\cdot \lVert w \rVert_0+b$, but $K(v) < K(w)$. 
\end{corollary}
\begin{proof}
	We show that the $\lVert\cdot\rVert_0$ norm is both properly additive (Definition \ref{def:properly-additive-vector-norms}) and super-logarithmic (Definition \ref{def:super-log-norm}).
	
	For the first part, let $A\subset \mathbb{N}$ be arbitrary.
	For any $j\in\mathbb{N}$ with $j\notin A$, we have
	\begin{align}
		\left\lVert \sum_{i\in A} e_i + e_j \right\rVert_0 &= |A|+1 > |A| = \left\lVert \sum_{i\in A} e_i \right\rVert_0.
	\end{align}
	
	For the second part, take $c=1,k=1,$ and choose $m_0=1$. For any $m\geq m_0$, it holds that
	\begin{align}
		\left\lVert \sum_{i=1}^{m} e_i \right\rVert_0 = m > \log_2(m).
	\end{align}
\end{proof}


\section{The Expressive Power of Recursion}
% TODO: Refer to G{\"o}del's primitive recursive functions. 
% \cite{godel1931formal}
% We can also take Robinson's version.

\section{Inductive Inference à la Solomonoff}
% TODO: Relate to Minimum Description Length Literature. 
% \cite{grunwald2007minimum} \cite{grunwald2019minimum}.
% TODO: Relate to Kolmogorov Complexity Literature. 
% \cite{li2008kolmogorov}.
% \cite{legg2007universal} Machine Intelligence via Kolmogorov Complexity.
By nature, our organism strives to minimize the energy it requires to perform a certain operation. This also applies to our brain. When trying to make sense of our world, our brain tries to do so in the most efficient way. But in terms of what kind of efficiency?

Let us illustrate this question with the example of inferring time series from few samples.
Observing the sequence \dots 2 \_ \_ \_ 2\dots, our brains might assume the constant function $f(n)=2$.
But given \dots 2 \_ 4 \_ \_ \dots, would the brain assume a linear sequence \dots 2 3 4 5 \dots? Or would it rather assume a constant function with one exception \dots 2 2 4 2 \dots.
And after the next sample extends the overall image to \dots 2 \_ 4 \_ 2\dots, is the constant function with one exception now still the most plausible assumption?
Or do we in fact deal with a symmetric piece-wise linear function \dots 1 2 3 4 3 2 1 \dots instead?

In general, both assumptions are reasonable, as they fit simple patterns on the observed sequences.
But as more and more 2s are joining the overall image, the constant function with the exception becomes more and more plausible. While it remains a suitable candidate for the underlying pattern, alternative pattern classes become more and more complex with additional samples, and thus more and more energy-intensive. Within our inductive bias that follows some principle of simplicity, such as Ockham's Razor, the simplest algorithm becomes more and more likely to generate the observed patterns. 
% TODO: Present Ockham's Razor more scientifically (with citation).

\subsection{The Necessity of Sequential Inputs in Alphabetical Representation}
%Humans do not perform operations on a number as a whole, as current mathematical models do. Instead, they represent numbers in numeral systems, in particular the decimal system. This transformation not only into a representation that allows
%Number is not regarded as a real number.

\subsection{Computational Complexity - The Yet Ignored Principle in Compression}
The term "simplicity" might hint at the \textit{descriptive} efficiency of an algorithm or concept. The less resources are needed to describe the algorithm, the more efficient will a machine or human be able to store this piece of information. 
However, this aspect does not fully capture the multi-sided shape of efficiency. Another side is the executive efficiency, that describes how many resources it needs to execute a certain algorithm. These resources include at least time, memory, but in a more general setting also communication costs between different involved units.

% TODO: Refer to Ray Solomonoff's Theory of Inductive Inference \cite{solomonoff1964formal}.

But given \dots 2 4 8 16 \dots, I do not assume a cubic polynomial, but an exponential function $f(n)=2^n$. Although the function values are exponential in the input, the computational complexity need not be, depending on which operations the underlying architecture allows. An architecture that features bit shift operations in constant time will allow an algorithm that computes $f$ with linear computational complexity.
Moreover, its descriptive complexity is far lower than the growing complexity of the polynomial alternatives.

% TODO: Example: Learning a sorting algorithm
 % !TEX root = thesis_ruettgers_lukas.tex
% From mitthesis package
% Version: 1.04, 2023/10/19
% Documentation: https://ctan.org/pkg/mitthesis

\chapter{Identifying the Simplest Consistent Algorithm}
We fix a universal reference machine $U$ and consider its induced enumeration of partial computable functions $f_1,f_2,\dots$.
Let $f:\Sigma^{*}\to\Sigma^{*}$ be an arbitrary, partial computable function over an arbitrary, but fixed, finite alphabet $\Sigma$.
For any $D\subseteq \Sigma^{*}$, we define
\begin{align}
	m_U(f\mid D)&:=\min_{\mathcal{S}\subset (D\times \Sigma^{*})^{*}}\{|S|\mid \text{The smallest } f_i \text{ consistent with } \mathcal{S} \text{ satisfies } f_i\equiv f\}, \text{ and}\\
	m_U(f)&:= \min_{D\subseteq \Sigma^{*}} m_U(f\mid D).
\end{align}
Analogously, we consider the amount of information that certainly suffices for all possible training domains and define $\mathcal{F}\subseteq 2^{\Sigma^{*}}$ as the collection of subsets of $\Sigma^{*}$ in which $f$ is still identifiable. That is, for any partial computable function $g\not\equiv f$, $g\lvert_D\not\equiv f\lvert_D$ for all $D\in \mathcal{F}$. Then, we define
\begin{align}
	M_U(f)&:= \max_{D\in\mathcal{F}} m_U(f\mid D).
\end{align}
\section{Out-Sampling Erroneous Simpler Algorithms}
What qualitative and quantitative criteria must the training sample meet to ensure that the simplest algorithm that is consistent with the sample truly coincides with the true function?

\section{Hypothesis Certification}
% see Chapter 5.2 in \cite{li2008kolmogorov}

\section{Retrievability of the Algorithm}
% \cite{richens2024robust} Nearly Optimal Policies for Local Interventions imply Retrievability of Causal Model.
We avail to the idea of \cite{richens2024robust} to derive sufficient conditions when the true algorithm is retrievable.

%% !TEX root = thesis_ruettgers_lukas.tex
% From mitthesis package
% Version: 1.04, 2023/10/19
% Documentation: https://ctan.org/pkg/mitthesis

%% ALTERNATIVE TITLE
% - Compression Algorithms in the Face of Kolmogorov Complexity
\chapter{Compression Algorithms against Kolmogorov Complexity}
\label{chap:compression-kolmogorov-complexity}

In Chapter \ref{chap:sufficient-information-learnability}, we formally established how Kolmogorov complexity smooths the way to quantify the information that a dataset of samples conveys about the functions that could have generated it.
Coined \textit{functional information}, this quantity allowed to derive general, yet nearly optimal, sufficient information-theoretic learnability conditions for virtually any partial computable function.
Moreover, we discussed how functional information not only helps to determine the relevant information content in a dataset, but also how it could realise a simplicity bias in practice that favours functions with lower Kolmogorov complexity.

But because it would evoke the same paradox as the halting problem, Kolmogorov complexity is not uniformly computable in general \cite{chaitin1974information}.
Notwithstanding that, heuristics could yield feasible approximations of this measure in practice.
Because we want to employ such a approximation algorithm to \textit{compare} the functional information of different objects, a necessary requirement for such an algorithm $A$ would be to yield at least some rough guarantee about the order between two strings $v,w$ in terms of their Kolmogorov complexity.
That is, there should be some exponential, multiplicative, and additive approximation offsets $a,b,k$ such that for any strings $v,w\in\{0,1\}^{*}$, if $A(v)\geq \exp_2^{(k)}\bigl(a\cdot A(w) + b\bigr)$, then $K(v)\geq K(w)$.
Here, $\exp_2^{(k)}$ designates the k-wise concatenation of the exponential function with basis $2$, just as in Definition \ref{def:recursive-concatenation}.

To date, compression algorithms are typically suggested to yield such a feasible approximation of Kolmogorov complexity.
Most importantly, several compression algorithms based on the classic Lempel-Ziv-Welch (LZW) algorithm \cite{welch1984technique} have been employed to cluster data of various formats \cite{cilibrasi2005clustering}.
However, this chapter proves that for any encoding of Turing Machines, the Lempel-Ziv-Welch algorithm can not yield any approximate guarantee about the order of the Kolmogorov complexity of two strings, rendering it at least theoretically undesirable for our purpose of approximating functional information.
In Theorem \ref{theorem:unbounded-order-inconsistencies-from-lzw-to-kolmogorov}, we prove that such approximation offsets $a,b,k$ as formulated above do not exist for the LZW algorithm.
This chapter therefore points up that practical implementations of a simplicity bias that yield at least approximate guarantees still need advances in compression algorithms.

But to that end, we first introduce the LZW algorithm in Section \ref{sec:limpel-ziv-welch-algorithm}, and state some required results about the compressibility of strings in Section \ref{sec:compressibility}.
In Section \ref{sec:unbounded-order-inconsistencies}, we eventually culminate in the aforementioned Theorem.

\section{The Lempel-Ziv-Welch algorithm}
\label{sec:limpel-ziv-welch-algorithm}
However, we substantiate that such methods still exhibit two essential shortcomings for our purpose to approximate functional information.
To that end, we first present how the Lempel-Ziv-Welch algorithm would actually compress binary strings.
Some adjustments have been made for the ease of exposition, but they do not affect our later argument.
\begin{algorithm}
	\caption{Lempel-Ziv-Welch algorithm \cite{welch1984technique}.}
	\label{alg:limpel-ziv-welch-algorithm}
	\small % Adjust font size
	\raggedright
	\renewcommand{\algorithmicrequire}{\textbf{Given:}} % Set name of subalgorithm
	\begin{algorithmic}[1]
		\REQUIRE{The binary alphabet $\Sigma=\{0,1\}$ and a non-empty string $x\in\{0,1\}^{*}$}
		\STATE{Instantiate a bidirectional key-value directory $T$ to save patterns. 
			\begin{enumerate}[label=(\alph*)]
				\item $T$ saves tuples $(\operatorname{id},v)$, where $v\in\{0,1\}^{*}$ is a pattern, and $\operatorname{id}\in\{0,1\}^{*}$ is its identifier.
				\item We write $T(v)=\begin{cases}
					\operatorname{id}, & (\operatorname{id},v)\in T\\
					\bot, & (\operatorname{id},v)\notin T \text{ for any }\operatorname{id}\in\mathbb{N}.
				\end{cases}$
				\item When a pattern $v$ is \textit{added} to $T$, the tuple $([|T|,\varepsilon],v)$ is added to $T$, where $|T|$ is the length of the directory $T$ and $[|T|,\varepsilon]$ its self-delimiting encoding. Once added, elements are not deleted. This way, the identifier is always unique.
				\item We add the symbols $0,1\in\Sigma$ as the elementary patterns to $T$. Therefore, it now holds that $(0,100),(1,101)\in T$, and $|T|=2$.
			\end{enumerate}
		}
		\STATE{Instantiate the empty output string $p:=\varepsilon$ that represents the compression of $x$.}
		\STATE{\textbf{While} $x$ is not in $T$ empty:
			\begin{enumerate}[label=(\roman{enumi})]
				\item Incrementally determine the shortest prefix $x_1x_2\cdots x_k$ of $x$, $x_i\in\{0,1\}$, that is not in $T$. 
				\item Append the identifier $T(x_1x_2\cdots x_{k-1})$ to $p$.
				\item Add the new pattern $x_1x_2\cdots x_k$ to $T$.
				\item Remove the prefix $x_1x_2\cdots x_{k-1}$ from $x$.
			\end{enumerate}
		}
		\STATE{Append the identifier $T(x)$ to $p$.}
		\RETURN{$p$, $T$}
	\end{algorithmic}
\end{algorithm}
%The length of $p$ could therefore be taken as the an approximation of $K(x)$, although there is no theoretical guarantee on the approximation factor.
The algorithm above merely looks for repetitions in the symbols sequences of a string.
The more often certain sequences of symbols occur in $x$, the smaller will the compression $p$ be.
However, this algorithm does not go further and consider the patterns \textit{across} the stored strings themselves.
Strings that are generated from quite simple recursive algorithm do however exhibit precisely such patterns, which the algorithm does yet not draw upon.

For this reason, the maximum compression ratio --- namely the ratio between the length of $n$ and its compression $p$ --- is bounded by a polynomial in the length of $n$, as the following Lemma ascertains.
\begin{lemma}[Bounded compression ratio Lempel-Ziv-Welch]
	\label{lemma:lzw-bounded-compression-ratio}
	Let $x\in\{0,1\}^{*}$ be an arbitrary string.
	Then the Lempel-Ziv-Welch compression $p$ that is output from Algorithm \ref{alg:limpel-ziv-welch-algorithm} satisfies $l(p)\geq\sqrt{l(x)}$.
\end{lemma}
\begin{proof}
	For an arbitrary $n\in\mathbb{N}$, consider the string $1^n$, that is the concatenation of $n$ $1$s.
	
	We show that this string achieves the optimal compression length that strings of length $n$ could possibly achieve.
	
	In each iteration, the prefix $x_1x_2\cdots x_{k}$ that is added to $T$ can at most be one symbol longer than the longest string in $T$.
	For if that was not the case, it holds that $k\geq 3$, because $0$ and $1$ are in $T$ from the beginning.
	But then $x_1x_2\cdots x_{k-1}$ is already not in $T$, which contradicts the assumption.
	For that reason, the optimal compression length is achieved if in each iteration, the prefix that is added achieves this maximal length.
	Because if such an $x_1x_2\cdots x_{k}$ is added to $T$, then $x_1x_2\cdots x_{k-1}$ is logarithmically compressed to $T(x_1x_2\cdots x_{k-1})$ and appended to $p$. 
	The longer this compressed sequence $x_1x_2\cdots x_{k-1}$ is, the shorter will $p$ be.
	
	Now, we show that $x=1^n$ achieves this optimal compression property.
	If $n=1$, then $p=1^1=1$, which is optimal.
	Below, we hence assume $n\geq 2$.
	In the first iteration $i=1$, $1$ is added to $p$ and $11$ is added to $T$.
	Therefore, $T$ contains a pattern string $v$ of length $i+1$ after iteration $i=1$.
	Now, assume that after some iteration $i=j$, $T$ contains all strings $1,11,\dots,1^{j+1}$.
	Then, in iteration $j+1$, if the remaining string $x$ is not longer than $j+1$, $T(x)$ is added to $p$, which is optimal since all prefixes $x_1x_2\cdots x_{k-1}$ that were added to $p$ had maximum length.
	Otherwise, if the remaining string $x$ is longer than $j+1$, then $1^{j+2}$ is the shortest prefix that is not in $T$, which is also optimal because there could be no such prefix that is longer.
	
	Now, let us define a lower bound on the length of $p$.
	Let $k$ be the smallest natural number such that $\sum_{i=1}^{k}i=\frac{k(k+1)}{2}\geq n$.
	Then, Algorithm \ref{alg:limpel-ziv-welch-algorithm} terminates after $k$ iterations, because in each iteration $i$ that is not the last one, $i$ symbols are removed from $x$.
	
	The string $T(x_1x_2\cdots x_{k-1})$ that is appended to $p$ in each such iteration obviously has a length of at least $1$.
	Because $\frac{(k+1)^2}{2}\geq n$, we also have $k\geq \sqrt{2n-1}$.
	Finally, since $n\geq 2$, $2n-1=n + (n-1)\geq n$. 
	By the monotonicity of the square root function, we therefore conclude
	\begin{equation}
		|p|\geq k \geq \sqrt{2n-1}\geq \sqrt{n} = \sqrt{(l(x))}.
	\end{equation}
\end{proof}

On the other hand, there is of course also an upper bound on the length of $p$.
Although the following upper bound is grossly conservative, it is sufficient for our needs and all the more simple to show.

\begin{lemma}[LZW Compression length upper bound]
	\label{lemma:lzw-compression-length-upper-bound}
	Let $x\in\{0,1\}^{*}$ be an arbitrary string.
	Then the Lempel-Ziv-Welch compression $p$ that is output from Algorithm \ref{alg:limpel-ziv-welch-algorithm} satisfies $l(p)\leq 3\cdot \bigl(l(x)\bigr)^2$.
\end{lemma}
\begin{proof}
	Let $x$ be an arbitrary string of length $n$.
	Because $0$ and $1$ are in $T$ from the beginning, the shortest prefix $x_1x_2\cdots x_{k}$ that is not in $T$ must have length $k\geq 2$.
	Therefore, at least one symbol is removed from $x$ in each iteration. Consequently, the LZW algorithm runs for at most $n-1$ iterations, since after $n-1$ iterations, the remaining $x$ will at most comprise one symbol, which is certainly in $T$.
	By the same argument as in the proof of Lemma \ref{lemma:lzw-bounded-compression-ratio}, $T$ can contain at most $n+1$ strings after iteration $n-1$, since $0$ and $1$ are added from the beginning.
	All in all, $p$ can at most contain $n$ compressions of strings whose identifying natural number is at most $n+1 -1 =n$.
	By the additional length of the self-delimiting encoding, we therefore conclude
	\begin{equation}
		l(p)=n\cdot (2\lceil\log_2(n)\rceil+1)\leq n\cdot (2n+1)=2n^2+n\leq 3n^2. 
	\end{equation}
\end{proof}

In contrast to the quadratic upper bound on the compression ratio of the Lempel-Ziv-Welch algorithm, the compression ratio that Turing Machines achieve are arbitrarily large, as the next Section illustrates.

\section{Compressibility of strings}
\label{sec:compressibility}
In the following, we fix $\Sigma=\{0,1\}$.
This section ascertains both the existence of incompressible strings and arbitrarily compressible strings, which smooths the way to Theorem \ref{theorem:unbounded-order-inconsistencies-from-lzw-to-kolmogorov} in the next section.
In the same vein as Lemma \ref{lemma:incompressible-strings-conditional}, we first argue that there must exist an incompressible string for each length $n$.
\begin{lemma}[Incompressible Strings]
	\label{lemma:incompressible-strings}
	For any $n\in\mathbb{N}$, there exists a string $v_n\in\{0,1\}^n$ with $K(v_n)\geq n$.
\end{lemma}
\begin{proof}
	Let $n\in\mathbb{N}$ be arbitrary.
	There are $2^n$ strings of length $n$, but only $\sum_{i=0}^{n-1}2^i=2^{n}-1$ strings that are shorter than $n$. 
	By the pigeonhole principle, there must be at least one $v\in\{0,1\}^n$ such that there exists no program $p$ of length $l(p)<n$ with $U(p)=v$.
	For this $v$, we have $K(v)\geq n$.
\end{proof}
In general, since the number of strings grows exponentially in their length, most strings are almost completely incompressible \cite{li2008kolmogorov}. 
Conversely, there are however also a few strings that are arbitrarily compressible.
While the Lempel-Ziv-Welch algorithm could not compress strings $z_n=1^n$ beyond a quadratic factor, there is a Turing Machine that achieves a compression ratio that scales faster than any polynomial, and even faster than any fixed concatenation of the exponential function, as the following Lemma certifies.

% Constructive Arbitary Complexity Upper Bound
\begin{lemma}[Arbitrarily Compressible Strings]
	\label{lemma:arbitrarycompress}
	Let $\operatorname{enc}(\cdot)$ be an arbitrary, prefix-free encoding of Turing Machines.
	Let $\exp_2^{(m)}$ denote the $m$-wise recursive concatenation of $\exp_2(x):=2^x$ as in Definition \ref{def:recursive-concatenation}.
	There exists a constant $c\in\mathbb{N}$ such that:
	For every $n\in\mathbb{N}$, there exists a string $w_n\in\{0,1\}$ of length $l(w_n)=\exp_2^{(n)}(1)$ with Kolmogorov complexity $K(w_n)\leq \log_2(n) + c$.
\end{lemma}
\begin{proof}
	Given everything as above, we construct a Turing Machine $\mathcal{T}$ that on input $x_n$ outputs the the string $w_n=(1)^{\exp_2^{(n)}(1)}$, which is the $\exp_2^{(n)}(1)$-wise concatenation of $1$. 
	Besides an input tape and an output tape, $\mathcal{T}$ uses one auxiliary tape.
	
	First of all, $\mathcal{T}$ asserts that the input $x\in\{0,1\}^{*}$ correctly encodes a natural number $n_x$ as by Definition \ref{def:natural-numbers-binary-strings-encoding}.
	Then, $\mathcal{T}$ writes a $1$ on the output tape and repeats the following procedure until the input tape only holds the empty string $\varepsilon$.
	
	%	The function $f$ computed by $\mathcal{T}$ will satisfy $f(x_n)=\begin{cases}
		%		1^{\left(\bigcirc_{i=1}^n \exp_2\right)(1)}, & n\geq 1\\
		%		1, & n=0.\\
		%	\end{cases}$
	\begin{itemize}
		\item Remove all non-blank symbols on the auxiliary tape.
		\item Copy the non-blank symbol sequence $x$ on the output tape to the auxiliary tape.
		\item Interpret $x$ as the natural number $n_x\in\mathbb{N}$. Compute $x_{2^{n_x}}$ and write it on the output tape.
		\item Subtract the number on the in
		put tape by $1$.
	\end{itemize}
	
	After this procedure, $\mathcal{T}$ copies $x$ once more to the auxiliary tape and eradicates the output tape blank. 
	Now, it repeats the following procedure as long as the string on the auxiliary tape is not blank, and terminates afterwards.
	\begin{itemize}
		\item Write a $1$ on the output tape.
		\item Subtract the string on the auxiliary tape by $1$.	
	\end{itemize}
	
	If $\mathcal{T}$ is invoked on $x_n$ for any $n\in\mathbb{N}$, the string on the output tape after the first iterative procedure encodes $\exp_2^{(n)}(1)$.
	After the second iterative procedure, we therefore have $(1)^{\exp_2^{(n)}(1)}$ written to the output tape as desired, and the universal Turing Machine $U$ from Section \ref{text:universal-tm} equally yields $(1)^{\exp_2^{(n)}(1)}$ when invoked on $\operatorname{enc}(\mathcal{T})x_n$.
	
	For that reason, choosing the constant as $c=l(\operatorname{enc}(\mathcal{T}))+1$, the Kolmogorov complexity of this string is bounded by
	\begin{align}
		K((1)^{\exp_2^{(n)}(1)})= K(\mathcal{T}(x_n)) = K\bigl(U(\operatorname{enc}(\mathcal{T})x_n)\bigr)
		\leq  l(\operatorname{enc}(\mathcal{T})) + l(x_n) \leq c+\log_2(n).
	\end{align}
	Finally, this string has length $\exp_2^{(n)}(1)$, which concludes our proof.
\end{proof}

These compressibility results already reveal the fundamental reason why the Lempel-Ziv-Welch algorithm and Kolmogorov complexity induce disparate orders on binary strings $v\in\{0,1\}^{*}$.
%Strings that are arbitrarily compressible by Turing Machines cannot be compressed beyond a polynomial factor by the LZW algorithm.
For sufficiently large $n$, there is a string $w_n$ of length $\exp_2^{(n)}(1)$ whose Kolmogorov complexity is smaller than that of an incompressible string $v_n$ of length $n$.
However, the compression length of the LZW algorithm cannot exceed a polynomial factor, rendering the compression of $w_n$ far larger than that of $v_n$. 
Since the exponential tower function $\exp_2^{(n)}(1)$ grows faster than any fixed concatenation of exponential functions,
Theorem \ref{theorem:unbounded-order-inconsistencies-from-lzw-to-kolmogorov} in the next section will demonstrate that this inconsistency even holds if we make arbitrarily large allowances for exponential, multiplicative, and additive approximation offsets.

\section{Unbounded order inconsistencies }
\label{sec:unbounded-order-inconsistencies}
On the one hand, Lemma \ref{lemma:lzw-bounded-compression-ratio} demonstrated that the maximum compression ratio of the LZW algorithm merely scales quadratically with growing length. 
Conversely, Lemma \ref{lemma:arbitrarycompress} demonstrated that strings such as $(1)^{\exp_2^{(n)}(1)}$ are arbitrarily compressible for increasing $n$, because their Kolmogorov complexity scales only logarithmically for increasing $n$.
No matter that prefix-free encoding of Turing Machines we choose, this discrepancy allows us to conclude that for any additive and multiplicative, and even exponential offset constants $a,b,k\in\mathbb{N}$, there are infinitely many strings $v,w$ such that 
\begin{equation}
	l\bigl(A(v)\bigr)\geq \exp_2^{(k)}\bigl(a\cdot l\bigl( A(w)\bigr) + b\bigr), \text{ but } K(v) < K(w),
\end{equation}
where $A$ is the LZW compression $p$ from Algorithm \ref{alg:limpel-ziv-welch-algorithm}.

But first and foremost, we concretize the dominance of exponential growth over polynomial growth by means of elementary calculus.
The proofs of the following two results are omitted here and laid out in Appendix \ref{sec:elementary-math} instead.
\begin{lemma}[Log-Linear Inequality with Additive Constant]
	\label{lemma:log-lin-add-inequality-placeholder}
	Let $a\geq 1,b\geq 0$ be arbitrary real numbers.
	For any real number $x > 2^{4(a+b)}$, it holds that $a\log_2(x)+b < x$.
\end{lemma}
The second Lemma ascertains the growth of the recursive exponentiation of a constant over linear functions.
\begin{lemma}[Special Cascade Exponential Equality]
	\label{lemma:cascade-exp-inequality-basis-placeholder}
	Let $\exp_2^{(m)}$ denote the $m$-wise recursive concatenation of $\exp_2(x):=2^x$ as in Definition \ref{def:recursive-concatenation}.
	Let $a,b$ be an arbitrary real scalars with $a\geq 1,b\geq 0$, and $k\in\mathbb{N}$. 
	For any natural number $n\in\mathbb{N}$ with $n>2^{k+2}\cdot (a+b)$, it holds that
	\begin{equation}
		\exp_2^{(n-k)}(1) > a\cdot n + b.
	\end{equation}
\end{lemma}

The aforementioned order inconsistencies more generally hold for any compression algorithm $A$ whose compression ratio is bounded by a fixed concatenation of exponential functions.
Therefore, even the aforementioned suggested proposal to recursively compress the pattern string dictionary would not resolve this issue, since the length of $p$ would still scale at least logarithmically with the length of $x$.
Because it even streamlines some technical details in the proof, we therefore prove this more general result in the following Theorem.
\begin{theorem}[Unbounded compression order inconsistencies]
	\label{theorem:unbounded-order-inconsistencies-from-lzw-to-kolmogorov}
	Let $\operatorname{enc}(\cdot)$ be an arbitrary, prefix-free encoding of Turing Machines.
	Let $A$ be an arbitrary compression algorithm such that there exist a length threshold $n_0$, an exponential compression ratio offset $m_0\in\mathbb{N}$, and a polynomial compression upper bound $p(x)=\sum_{i=0}^{k_0}a_ix^i$ such that for all inputs $x$ with length $l(x)\geq n_0$, it holds that $\log_2^{(m_0)}\bigl(l(x)\bigr) \leq l\bigl(A(x)\bigr)\leq p\bigl(l(x)\bigr)$.
	
	Then, for any real numbers $a\geq 1,b\geq 0$, and any $k\in\mathbb{N}$,
	there are strings $v,w\in\{0,1\}^{*}$, such that
	\begin{equation}
		l\bigl(A(v)\bigr)\geq \exp_2^{(k)}\bigl(a\cdot l\bigl( A(w)\bigr) + b\bigr), \text{ but } K(v) < K(w).
	\end{equation}
\end{theorem}
\begin{proof}
	Given $\operatorname{enc}(\cdot)$, the compression algorithm $A$ with $n_0,m_0,p(x)=\sum_{i=0}^{k_0}a_ix^i$, and the constants $a,b,k$ as above.
	Let $c$ be the constant from Lemma \ref{lemma:arbitrarycompress}.
	Fix an arbitrary $n\in\mathbb{N}$ with $n>\max\Bigl(2^{4 (1+c)},n_0,2^{k+m_0+3}\cdot\bigl(k_0+a\cdot \sum_{i=0}^{k_0}a_i + b\bigr)\Bigr)$.
	
	On the one hand, Lemma \ref{lemma:incompressible-strings} guarantees the existence of an incompressible string $v_n\in\{0,1\}^{*}$ with $K(v_n)\geq n$.
	On the other hand, we consider the compressible string $w_n=1^{\exp_2^{(n)}(1)}$ from Lemma \ref{lemma:arbitrarycompress} with $K(z_n)\leq \log_2(n)+c$.
	
	%% 1. KOLMOGOROV INEQUALITY
	\textbf{1. Kolmogorov Complexity Inequality}
	
	Because $n>2^{4(1+c)}$, Lemma \ref{lemma:log-lin-add-inequality-placeholder} asserts that 
	\begin{align}
		K(v_n)\overset{\ref{lemma:incompressible-strings}}{\geq} n \overset{\ref{lemma:log-lin-add-inequality-placeholder}}{>} \log_2(n)+c \overset{\ref{lemma:arbitrarycompress}}{\geq} K(w_n).
	\end{align}
	
	%% 2. LEMPEL-VIZ-WELCH INEQUALITY
	\textbf{2. Compression Inequality}
	
	Since $n>n_0$, the compression ratio bound of $A$ ensures that
	\begin{align}
		\label{eq:theorem-arbitrary-order-inconsistency-avn-to-exp-m0}
		l\bigl(A(v_n)\bigr)\geq \log_2^{(m_0)}\bigl(l(v_n)\bigr) = \log_2^{(m_0)}\bigl(\exp_2^{(n)}(1)\bigr)=\exp_2^{(n-m_0)}(1).
	\end{align}
	
	Because $2^n>n$ and $n\geq 1$, we also have
	\begin{align}
		\label{eq:theorem-arbitrary-order-inconsistency-exp-to-p}
		2^{k_0n+a\cdot \sum_{i=0}^{k_0}a_i + b} = \bigl(2^{a\cdot \sum_{i=0}^{k_0}a_i + b}\bigr)\bigl(2^{n}\bigr)^{k_0}>\Bigl(a\cdot \sum_{i=0}^{k_0}a_i + b\Bigr) n^{k_0} \geq a\cdot \Bigl( \sum_{i=0}^{k_0}a_in^{i} \Bigr) + b. 
	\end{align}
	
	Because $n>2^{k+m_0+3}\cdot\bigl(k_0+a\cdot \sum_{i=0}^{k_0}a_i + b\bigr)$, Lemma \ref{lemma:cascade-exp-inequality-basis-placeholder} also guarantees that 
	\begin{align}
		\label{eq:theorem-arbitrary-order-inconsistency-exp-to-lin}
		\exp_2^{(n-k-m_0-1)}(1)> k_0n+a\cdot \sum_{i=0}^{k_0}a_i + b.
	\end{align}
	
	By the strict monotonicity of $\exp_2$, we therefore conclude our proof with
	\begin{align}
		l\bigl(A(v_n)\bigr)&\overset{(\ref{eq:theorem-arbitrary-order-inconsistency-avn-to-exp-m0})}{\geq}\exp_2^{(n-m_0)}(1)\\
		&= \exp_2^{(k)}\bigl(\exp_2^{(n-k-m_0)}(1)\bigr) \\
		&= \exp_2^{(k)}\Bigl(\exp_2\bigl(\exp_2^{(n-k-m_0-1)}(1)\bigr)\Bigr) \\
		&\overset{(\ref{eq:theorem-arbitrary-order-inconsistency-exp-to-lin})}{>} \exp_2^{(k)}\Bigl(\exp_2\bigl(k_0n+a\cdot \sum_{i=0}^{k_0}a_i + b\bigr)\Bigr)\\
		&\overset{(\ref{eq:theorem-arbitrary-order-inconsistency-exp-to-p})}{>}\exp_2^{(k)}\Bigl( a\cdot \bigl( \sum_{i=0}^{k_0}a_in^{i} \bigr) + b \Bigr)\\
		&\geq \exp_2^{(k)}\Bigl( a\cdot l\bigl(A(w_n)\bigr) + b \Bigr).
	\end{align}
\end{proof}
The Lempel-Ziv-Welch compression from Algorithm \ref{alg:limpel-ziv-welch-algorithm} certainly satisfies these bounded compression ratio conditions.
On the one side, Lemma \ref{lemma:lzw-compression-length-upper-bound} allows us to choose $p(x)=3x^2$ as an upper bound on the compression length.
On the other side, since any $n>2^{16}$ satisfies $\sqrt{n}>\log_2(n)$ by virtue of Lemma \ref{lemma:log-lin-add-inequality-placeholder}, the compression length lower bound in Lemma \ref{lemma:lzw-bounded-compression-ratio} renders $\log_2(n)$ a sufficient lower bound with $n_0=2^{16}$.
But even more modern compression algorithms like the DEFLATE algorithm \cite{deutsch1996rfc1951}, which is widely adopted in file compression, or the Brotli compression \cite{alakuijala2018brotli}, which recently exhibited experimental superiority over the so far unchallenged DEFLATE algorithm \cite{alakuijala2015comparison}, would not yield satisfactory approximations of the Kolmogorov complexity order between two strings.
Furthermore, although it sufficed to neglect it for the above proof, the length of the additional pattern dictionary $T$ must also be taken into account when quantifying the compression ratio, because $x$ cannot be restored from $p$ without $T$.

But even if compression algorithms were to yield such rough guarantees, the implementation of functional information would still require some further adjustments to such algorithms.
While compression algorithms usually compress data ad hoc without prior knowledge, functional information instead conditions on the information of the given instances $x_i$. 
In this vein, we merely require to find the shortest prefix $p$ such that $U(px_i)=y_i$.
Moreover, functional information imposes \textit{multiple} constraints on $p$, because $p$ needs to satisfy $U(px_i)=y_i$ for all samples $(x_i,y_i)$ in the dataset $D$.
The larger $D$ becomes, the harder will this constrained optimization problem appear too.

If the second requirements renders too difficult to realise, it might hence also be acceptable to work with the \textit{joint} functional information $K_{JF}(D)$, as this definition simply concatenates the labels and instances to one long string.
Although this comes at the cost of spoiling theoretical guarantees, it might yield satisfactory approximations in practice.

This discussion stresses how the problem of learning a simplest consistent function is deeply intertwined with compression.
An algorithm that would be able to compress the causal mechanism that determines $y_i$ given $x_i$ into a small string $p$ directly tackles the learning problem itself.
The recent advances in compression theory \cite{campi2023compression} could therefore impact both the theory of generalization and practical algorithms in machine learning.




%%% Appendicies of thesis  %%%%%%%%%%%%%%%%%%%%%%%%%%%%%%%%%%%%%%%%%%%%%%%%%%%%%%%%%%%%%%%%%%%%%%%%

\appendix
% !TEX root = thesis_ruettgers_lukas.tex
% From mitthesis package
% Version: 1.01, 2023/07/04
% Documentation: https://ctan.org/pkg/mitthesis

\chapter{Secondary Theory and Proofs}

\lstdefinestyle{mystyle}{
	backgroundcolor=\color{CadetBlue!15!white},   
	commentstyle=\color{Red3},
	numberstyle=\tiny\color{gray},
	stringstyle=\color{Blue3},
	basicstyle=\small\ttfamily,
	breakatwhitespace=false,         
	breaklines=true,                 
	numbers=left,                    
	numbersep=5pt,                  
	showspaces=false,                
	showstringspaces=false,
	showtabs=false,                  
	tabsize=2
}%
\lstset{language=C++,style={mystyle}}%
\section{Elementary calculus}
\label{sec:elementary-math}

\begin{lemma}[Log-Linear Inequality]
	\label{lemma:log-lin-inequality}
	Let $a\geq 1$ be an arbitrary real number.
	For any real number $x> 2^{4a}$, $a\log_2(x)< x$.
\end{lemma}
\begin{proof}
	Fix an arbitrary real number $a\geq 1$.
	
	First of all, we show that $x<2^x$ for all $x\in\mathbb{R}$ with $x\geq 1$ by elementary calculus.
	Define the real functions $g(x)=x,h(x)=2^x$.
	Both $g$ and $h$ are continuous differentiable with derivatives $g'(x)=1,h'(x)=\ln(2)2^x$.
	For any $x\geq 1$, the strict monotonicity of the exponential function yields
	\begin{equation}
		\label{eq:exp-lin-derivative-inequality}
		h'(x)=\ln(2)2^x\geq 2\ln(2)=\ln(4)\geq \ln(e)=1=g'(x).
	\end{equation}
	But we also have $h(1)=2>1=g(1)$.
	Therefore, for any $x\geq 1$, it holds that
	\begin{equation}
		\label{eq:exp-lin-inequality}
		2^x=h(x)=h(1)+\int_{1}^{x}h'(x')dx'\overset{(\ref{eq:exp-lin-derivative-inequality})}{>} g(1)+\int_{1}^{x}g'(x')dx'=g(1)+g(x)-g(1)=g(x).
	\end{equation}
	
	In particular, the above result implies $\log_2(x)< x$ for all $x\geq 1$ by substituting $x=2^z$ into Equation \ref{eq:exp-lin-inequality}.
	To generalize this argument to $a\log_2(n)$, we conduct an analogous argument for the real functions $g_a(x)=a\log_2(x)$ and $h(x)=x$.
	
	By $\log_2(x)=\log_2(e)\ln(x)$ for all $x>0$, we obtain the derivative of $g_a$ as $g_a'(x)=\log_2(e)a\frac{1}{x}$.
	Since $\log_2(e) < 2$, any $x\geq 2a$ satisfies
	\begin{equation}
		\label{eq:log-lin-factor-derivative-inequality}
		g'_a(x)=\log_2(e)a\frac{1}{x}< 2a\frac{1}{2a}=1=h'(x).
	\end{equation}
	
	Now, take $x=2^{4a}$. Since $2^{2a} \geq 2^2 = 4$ by the assumption $a\geq 1$, it holds that
	\begin{equation}
		\label{eq:log-lin-factor-inequality-init}
		g_a(x)=a\log_2(x)=a\log_2(2^{4a})=4\cdot a \cdot a \overset{(\ref{eq:exp-lin-inequality})}{<}2^{2a}\cdot 2^a\cdot 2^a=2^{4a}=x=h(x).
	\end{equation}
	
	As $2^{4a}\geq 2^{2a}\geq 2a$, for all $x> 2^{4a}$ this eventually yields
	\begin{align}
		\label{eq:log-lin-inequality}
		x=h(x)&=h(2^{4a})+\int_{2^{4a}}^{x}h'(x')dx'\\
		&\overset{(\ref{eq:log-lin-factor-derivative-inequality})}{>} g_a(2^{4a})+\int_{2^{4a}}^{x}g_a'(x')dx'\\
		&=g_a(2^{4a})+g_a(x)-g_a(2^{4a})=g_a(x)=a\log_2(x).
	\end{align}
\end{proof}

\begin{corollary}[Log-Linear Inequality with Additive Constant]
	\label{cor:log-lin-add-inequality}
	Let $a\geq 1,b\geq 0$ be arbitrary real numbers.
	For any real number $x> 2^{4(a+b)}$, $a\log_2(x)+b < x$.
\end{corollary}
\begin{proof}
	Because $a\geq 1$, it is guaranteed that $x\geq 2^4 \geq 2$, and hence $\log_2(x)\geq 1$. 
	Since $a+b\geq 1$, we employ Lemma \ref{lemma:log-lin-inequality} to conclude $a\log_2(x)+b < (a+b)\log_2(x)\leq x$.
\end{proof}

\begin{lemma}[General Cascade Exponential Inequality]
	\label{lemma:cascade-exp-inequality}
	Let $\exp_2^{(m)}$ denote the $m$-wise recursive concatenation of $\exp_2(x):=2^x$ as in Definition \ref{def:recursive-concatenation}.
	Let $a$ be an arbitrary real scalar with $a\geq 1$, and $k\in\mathbb{N}$. 
	For any natural number $n\in\mathbb{N}$ with $n>4\cdot 2^k\cdot a$ and any real number $x\geq 2$, it holds that
	\begin{equation}
		\exp_2^{(n-k)}(x) > a\cdot n \cdot x.
	\end{equation}
\end{lemma}
\begin{proof}
	% TODO: Replace $\bigcirc$ notation by the $m$-wise recursive concatenation.
	Denote by $g_2$ the real function $g_2(x)=2x$.
	A similarly elementary argument as in the proof of Lemma \ref{lemma:log-lin-inequality} yields $\exp_2(x)=2^x\geq 2x=g_2(x)$ for any $x\geq 2$.
	Repeated application of this inequality thence yields 
	\begin{equation}
		\label{eq:multi-exp-2x-inequality}
		\left(\bigcirc_{i=1}^n\exp_2\right)(x) \geq \left(\bigcirc_{i=1}^n g_2\right)(x)= 2^n\cdot x \quad \text{ for any } n\in\mathbb{N}.
	\end{equation}
	
	In particular, this states that $\left(\bigcirc_{i=1}^{n-k}\exp_2\right)(x) \geq 2^{n-k}\cdot x$ for any $n\geq k$.
	
	Now, assume that $n>4\cdot 2^k\cdot a$.
	By Lemma \ref{lemma:log-lin-inequality}, it holds that $2^n> 2^k\cdot a\cdot n$.
	For that reason, we obtain
	\begin{align}
		\label{eq:cascade-exp-inequality}
		\left(\bigcirc_{i=1}^{n-k}\exp_2\right)(x) &\overset{(\ref{eq:multi-exp-2x-inequality})}{\geq} 2^{n-k}\cdot x\\
		&\overset{\ref{lemma:log-lin-inequality}}{>} 2^{-k}\cdot 2^k\cdot a\cdot n \cdot x = a\cdot n\cdot x.
	\end{align}
	
\end{proof}
\begin{corollary}[Special Cascade Exponential Equality]
	\label{cor:cascade-exp-inequality-basis}
	Let $a\geq 1,b\geq 0$ be arbitrary real numbers.
	For any $k\in\mathbb{N}$ and $n\in\mathbb{N}$ with $n>4\cdot 2^k\cdot (a+b)$, it holds that
	\begin{equation}
		\left(\bigcirc_{i=1}^{n-k}\exp_2\right)(1) > a\cdot n + b.
	\end{equation}
\end{corollary}
\begin{proof}
	Let $a\geq 1,b\geq 0$ be arbitrary real numbers.
	Let $n>4\cdot 2^k \cdot (a+b)$ be arbitrary. 
	First of all, Equation \ref{eq:exp-lin-inequality} asserts that $2^k>k$.
	With $a+b\geq 1$, this implies $n>4\cdot k\geq k+1$ for $k\geq 1$.
	For $k=0$, we similarly have $n>4\cdot 2^k=4>k+1$.
	Therefore, we may reduce $\left(\bigcirc_{i=1}^{n-k}\exp_2\right)(1)=\left(\bigcirc_{i=1}^{n-k-1}\exp_2\right)(2)$ and conclude in the same fashion as Lemma \ref{lemma:cascade-exp-inequality}:
	\begin{align}
		\left(\bigcirc_{i=1}^{n-k-1}\exp_2\right)(2) &\overset{(\ref{eq:multi-exp-2x-inequality})}{\geq} 2^{-k-1}\cdot 2^n \cdot 2\\
		&\overset{(\ref{eq:cascade-exp-inequality})}{>} 2^{-k-1} \cdot 2^k\cdot (a+b)\cdot n \cdot 2\\
		& = (a+b)\cdot n \overset{n\geq 1}{\geq} an+b.
	\end{align}
\end{proof}
\section{Inexpressivity of scaled recursive completion}
\label{app:scaled-recursive-completion-inexpressible}
\begin{corollary}[Scaled Recursive Completion is not non-recursively expressible]
	\label{cor:scaled-recursive-completion-not-non-recursive-expressible}
	Let $\tau$ be as in Theorem \ref{theorem:recursive-completion-not-non-recursive-expressible}.
	
	For any $a\in\mathbb{N}$, define the scaled recursive completion over $\tau$ as
	$\left(f_{\tau}\right)_{a}^{\lozenge}(n):=f_{\tau}^{(n)}(a\cdot n)$.
	
	Then, $\left(f_{\tau}\right)_{a}^{\lozenge}\notin \mathcal{F}_{\tau}$ for any $a \geq 1$.
	Let $f\in\mathcal{F}_{\tau}$ be an arbitrary non-recursive function over $\tau$.
	Then, for the same $n_0(f)$ as in Theorem \ref{theorem:recursive-completion-not-non-recursive-expressible}, we have that for all $a\in\mathbb{N}$ with $a\geq 1$ and $n\geq n_0$, $f(n)\neq \left(f_{\tau}\right)_{a}^{\lozenge}(n)$ for all $n\geq n_0(f)$.
\end{corollary}
\begin{proof}
	We show that the thresholds $n_0$ in the proof of Theorem \ref{theorem:recursive-completion-not-non-recursive-expressible} maintain to hold for $\left(f_{\tau}\right)_{s}^{\lozenge}$ for any scale $s\in\mathbb{N}_{\geq 1}$.
	In the following, let $s\in\mathbb{N}$ be arbitrary with $s\geq 1$.
	
	Analogously to Theorem \ref{theorem:recursive-completion-not-non-recursive-expressible}, we prove a stronger argument by induction over the depth of non-recursive functions.
	That is, we show that for all non-recursive functions $f\in\mathcal{F}_{\tau}$, there is an $n_0\in\mathbb{N}$ such that $f(n+a)<f_{\tau}^{(n)}(s\cdot(n+a))$ for all $n\geq n_0$ and all offsets $a\in\mathbb{N}$.
	Note that this statement was already proved for $s=1$ in the original theorem.
	
	Since the assumptions on $\tau$ are the same as in Theorem \ref{theorem:recursive-completion-not-non-recursive-expressible}, we can directly avail to its equations.
	For depth $0$ and any $n\geq n_0=1,a\in\mathbb{N}$, we extend Equations \ref{eq:theorem:recursive-completion-not-non-recursive-expressible-self-lower-bound} and \ref{eq:theorem:recursive-completion-not-non-recursive-expressible-inequality} to
	\begin{align}
		f_{\tau}^{(n)}(s\cdot (n+a)) &\overset{(\ref{eq:theorem:recursive-completion-not-non-recursive-expressible-inequality})}{\geq} f_m^{(n)}(s\cdot (n+a))\\
		&\geq f_m^{(n)}(n+a)\\
		\label{eq:cor-scaled-recursive-completion-not-non-recursive-expressible-self-lower-bound}
		&\overset{(\ref{eq:theorem:recursive-completion-not-non-recursive-expressible-self-lower-bound})}{>}n+a  = f(n+a).
	\end{align}
	
	Proceeding with the induction hypothesis (IH), assume that there is some $p\in\mathbb{N}$ such that for every non-recursive function $f\in\mathcal{F}_{\tau}$ with $\operatorname{dep}(f)\leq p$, there exists an $n_0\in\mathbb{N}$ such that $f_\tau^{(n)}(s\cdot(n+a))>f(n+a)$ for all $n\geq n_0$ and $a\in\mathbb{N}$. 
	As in Theorem \ref{theorem:recursive-completion-not-non-recursive-expressible}, we expand any $f\in\mathcal{F}_{\tau}$ with depth $p+1\geq 1$ as $f(n)=f_m(g_1(n),\dots,g_k(n))$ for some $f_m\in\tau$ with arity $k\in\mathbb{N}$ and non-recursive functions $g_i\in \mathcal{F}_{\tau}$ with depth $\operatorname{dep}(g_i)\leq p$ and distinguish by the cases where $f_m$ is bounded or strictly monotonously increasing.
	
	In the case that $f_m$ is bounded by some $c_m$, the same $n_0=c_m$ as in Equation \ref{eq:theorem:recursive-completion-not-non-recursive-expressible-induction-step-bounded} in the proof of the original theorem satisfies $f_{\tau}^{(n)}(s\cdot (n+a)) \overset{(\ref{eq:cor-scaled-recursive-completion-not-non-recursive-expressible-self-lower-bound})}{>} n+a \geq c_m \geq f(n+a)$.
	
	In the other case where $f_m$ is strictly monotonously increasing, we adopt $n_0=\max_{1\leq i\leq k} n_0(g_i)$ as it stands from the proof of Theorem \ref{theorem:recursive-completion-not-non-recursive-expressible}.
	Similarly, let $n\geq n_0$ and $a\geq 1$ be arbitrary and substitute $n'=n+1$ and $b=a-1$. Then we analogously have
	\begin{align}
		f(n'+b)=f(n+a)&=f_m(g_1(n+a),\dots,g_k(n+a))\\
		&\overset{(IH)}{<}f_m(f_\tau^{(n)}(s\cdot (n+a)),\dots,f_\tau^{(n)}(s\cdot (n+a)))=f_m^{-}(f_\tau^{(n)}(s\cdot (n+a)))\\
		&\overset{\ref{lemma:max-bound-recursive-concatenation-sum}}{\leq} f_\tau^{-}(f_\tau^{(n)}(s\cdot (n+a)))=f_{\tau}^{(n+1)}(s\cdot (n+a))=f_{\tau}^{(n')}(s\cdot(n'+b)).
	\end{align}
	
	The overall statement consequently follows by the induction principle, and as a special case, we obtain that for any non-recursive function $f\in\mathcal{F}_{\tau}$, there exists an $n_0(f)\in\mathbb{N}$ such that $f(n)<f_{\tau}^{(n)}(s\cdot n)=\left(f_{\tau}\right)_{s}^{\lozenge}(n)$ for all $n\geq n_0$.
	
	This $n_0$ does not depend on $s$ and applies for all $s\geq 1$, which proves our result.
	
\end{proof}
%\include{appendixb}


%%% Bibliography  %%%%%%%%%%%%%%%%%%%%%%%%%%%%%%%%%%%%%%%%%%%%%%%%%%%%%%%%%%%%%%%%%%%%%%%%%%%%%%%%%

\printbibliography[title={References},heading=bibintoc]

% biblatex also supports chapter-by-chapter bibliography, https://tex.stackexchange.com/a/296502/119566
% see the biblatex manual, section 3.14.3


%%%% Option for natbib %%%%%%%%%%%%%

%%   use an appropriate style (.bst) and your own .bib file[s]

%\bibliographystyle{plainnat}
%\bibliography{mitthesis-sample.bib}

\end{document} 
 